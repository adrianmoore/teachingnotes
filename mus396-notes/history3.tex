%\chapter{USSS toolkits}
%shown in figure~\ref{fig:artificialnatural}
%pages~\pageref{tab:opposites1}
%and form (section~\ref{section:form}, page~\pageref{section:form}).

\chapter{Electronic music 2}
\label{history3}

\section{Synthesizers and Computers}
\begin{itemize}
\item History - Hugh le Caine and Oskar Sala
\item Switched on Bach - Wendy Carlos
\item Computers
\item The Institutions - Bell labs.
\item Matthews, Tenney and others
\item The history of the Music N languages
\item CSOUND
\end{itemize}

Two more important (if unrecognized names) in history.

\href{http://www.hughlecaine.com}{Hugh Le Caine}, scientist who worked on the development of radar during WWII and afterwards nuclear physics. Best known for the design of an Electronic Sackbut (1945), The National Research Council, Ottawa established a research studio for him. He also designed a multi-track tape recorder (1953). One of his best known works is \textit{Dripsody} (OHM CD1 track 07) (1955) based upon the sound of a drop of water.

Oskar Sala, the German composer and physicist whose novel musical instrument produced the sound effects for Alfred Hitchcock's "The Birds," died Tuesday, February 26 2002. He was 91. Born in the eastern German town of Greiz, Sala is known for developing and mastering the trautonium, billed as the world's first electronic musical instrument on its invention in 1929. He performed with the Berlin Philharmonic several times, and the instrument - a precursor to the synthesizer - was frequently used in German advertss in the 1940s and 1950s.

The trautonium was most famously employed to produce the bird calls in Hitchcock's 1963 film. Few people realized the cacophonous calls on the film were produced electronically. Sala donated his original Mixtur-Trautonium to the German Museum for Contemporary Technology in Bonn in 1995. Example \textit{concertando rubato from elektronische tanzuite} - 1955/1989 (OHM- CD1, track9)


\subsection{Bell Laboratories and the development of Music N languages}

1948 - Dr. Werner Meyer-Eppler, director of the department of Phonetics at Bonn University was visited by Homer Dudley, researcher at Bell Telephone Labs, New Jersey, USA. Dudley demonstrated the Vocoder (Coice Operated reCOrDER).

Ussachevsky and Luening also discovered research at Bell Telephone laboratories, New Jersey into computer analysis and (re)synthesis of sounds.

\begin{quotation}
The work at Bell Telephone Laboratories proved to be unique. Apart from Ampex, who were involved in the desin and manufacture of recording and amplification equipment, most industrial concerns were not willing even to consider supporting reasearch and development in this field unless they could expect to benefit commercially from the results within a matter of months. \citep[94]{manning2013electronic}
\end{quotation}


Bell Labs, in Murray Hill, New Jersey. Researchers: Max Mathews and John Pierce.

In 1957 Mathews finished Music 1. The first piece produced with MUSIC 1 was a 17second composition by Newman Guttman called The Silver Scale (1957). Mathews - ``it was terrible''. One waveform, a triangle wave. Control over pitch, loudness and duration. Then came numerous revisions of the software – hence the term Music N languages. Music III used the instrument idea and the concept of score and orchestra.

Pierce hired James Tenney who worked in Bell Labs between 1961 and 1964. \textit{Dialogue} (1963) is a study in tone vs. noise.

Music IV was finished – in 1962.

Jean-Claude Risset was doing graduate work in physics at the Ecole Normale Sup\'erieur in Paris but received money to work at Bell in 1964 as research composer in residence. Research: Investigation of instrument tones. (An introductory catalogue of computer synthesized sounds - Risset, 1969)

John Chowning got hold of a copy of the MUSIC IV cards.

The sounds we could make with a computer at that time were unbearably dull, not because we couldn't do better but we didn't know a lot. So in 1967, with ears starved for some sound that had the richness of the sounds we hear in nature, I was experimenting with extreme vibrato. I realized that as I increased the rate and depth of a vibrato, I was no longer hearing it as a change in pitch, but rather as a change in timbre. It was a kind of frequency modulation.
1971 – Chowning hit on the FM theory for good. But few were interested. "the the office of licensing contacted Yamaha, and Yamaha sent an engineer. I played some examples, and in ten minutes he understood." Chowning Patented

Paper published in the Journal of the Audio Engineering Society, September 1973. Research was strong at that time especially with the design and commission of the Samson box, based programmed via a PDP-10 and then functioned as a 'realtime MUSIC IV'. Much money was sifted to this research and Chowning formed CCRMA (Center for Computer Research in Music and Acoustics)…."to have some basis on which we could get more grants." CCRMA became (and still perhaps is) the primary research center. Chowning (Turenas, 1972)

In 1969 JCR moved back to France whereupon he met with Boulez who wanted him to head up the computer department of the newly formed IRCAM. Max Mathews was scientific director at the beginning. Official opening of IRCAM in1977.

Risset - endless glissando (track 40) / music (\textit{mutations}, 1969).

