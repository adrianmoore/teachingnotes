%\chapter{USSS toolkits}
%shown in figure~\ref{fig:artificialnatural}
%pages~\pageref{tab:opposites1}
%and form (section~\ref{section:form}, page~\pageref{section:form}).

\chapter{Electronic music 5}
\label{history5}

\section{Computer aided composition and commercialism, MIDI and Sampling}

\textit{The digital revolution.}

Outline of the computer. Programming. Machine language was numeric, Assembler used codes(a large list of mnemonics). But still some higher form of language was required (but this slows down processes) because Assembler was often machine dependant and very difficult to learn or use.  Higher level languages then employed a simpler instruction set where each command corresponded to multiple machine instructions.  Programs included ALGOL (ALGOrithmic Language), BASIC, C, COBOL, FORTRAN (FORmula TRANslation), LISP, and PASCAL are common today.  These languages are then compiled to code often, very efficiently.  

Bell labs experiments and Max Mathews. The Music X languages.  Max Mathews was a Bell Labs engineer used the computer to explore sound synthesis with MUSIC I in 1957 and MUSIC II in 1958.  The computer he used was an IBM 704 based upon Valve technology.   When Bell Labs took possession of one of the first new transistor based IBMs, a 7094 in 1959, Mathews produced MUSIC III (1960).

MUSIC IV appeared in 1962, the code still being written in assembler.

Interest in Mathews' project from James Tenney.  As the IC computers emerged in the mid 60s, with the IBM360 Models the portability of software became more important.  Hence the first translation to Fortran in 1965.  Mathews completed MUSIC V in 1968 (all in FORTRAN) but now others were redeveloping the code, including Barry Vercoe at Princeton who created MUSIC 360 (for the ibm machines). Then came MUSIC 11 for the PDP 11 computers in 1973, and MUSIC 10 (for the PDP 10s) written by John Chowning and James Moorer at the University of Stanford, California in 1975.

As C was becoming the language of choice Richard Moore developed CMUSIC, a much expanded version of MUSIC V in 1985.  In a similar move, Barry Vercoe translated MUSIC 11 into a C version known at CSOUND at MIT in 1986.

During the 1970s John Chowning  developed FM synthesis which was then incorporated into the MUSIC systems.  His works in 1972, \textit{Sabelithe} and \textit{Turenas}(anagram of natures) use the technique extremely well.

\subsection{Additive and Subtractive synthesis}

As Manning points out the experimental nature of the early works of BELL Labs were shown in the rudimentary style of work and the basic titles such as \textit{Variations in Timbre and Attack} (1961) and \textit{Five Against Seven – Random Cannon} (1961) by John Pierce, or \textit{Noise Study} (1961) or \textit{Five Stochastics Studies} (1961) by James Tenney.  \textit{Bicycle Built for Two} (1962) – very early vocal synthesis achieved fame through reincarnation in the 2001 film….as it is referred to by Hal as he reverts to type. 

Jean-Claude Risset. Worked at BL for a time with very successful pieces such as \textit{Little Boy} (1968), \textit{Dialogues} (1975) for flute, clarinet, piano, percussion and computer-generated sounds, and \textit{Songes} (1978), featuring transformations from violin and flute sounds to the complex timbres of frequency-modulated tones.  Risset then took this work to IRCAM as he was made head of computer facilities there. 

In the UK, Jonathan Harvey must be mentioned as a composer working along similar lines as Risset. He studied in Princeton 1969-70 and made there the piece Time Points.  It is for \textit{MPVV} that he is best remembered (SEE EMMERSON)

In the synthesiser market however, Peter Zinovieff created a large establishment in London in the 70s.  A hybrid system called MUSYS III was developed. Zinovieff then created a company called EMS.  His studio began back in 1962 with basic voltage control.  He constructed a sequencer in 1966-8.  PDP computers were purchased in 1969, software constructed and MUSYSIII was ready.  Synths like the VCS4 were replaced by the Synthi 100 and EMS went very commercial.  With this system (MUSYSIII), Birtwistle composed Chronometer in 1971 (made from clock sounds).




\subsection{Early Computer Aided Composition} 
Pioneers were Xenakis and Lejaren Hiller – Hiller for his work with cage on \textit{HPSCHD} (1967-9)

Xenakis – New proposals in microsound structure – found in Formalized Music – 1968.

`only when the pure electronic sounds were framed by other `concrete' sounds which were much richer and much more interesting (thanks to Varese, Schaeffer and Henry) could electronic music become really powerful.'


Early work included ST/10-1, 080262, a work for instrumental ensemble, received its first performance on 24 May 1962 at the headquarters of IBM France.  Generated were…time of occurrence of event, type of timbre, choice of instrument, gradient of glissando, the duration and dynamic. Other works include some multimedia pieces \textit{Polytope de Cluny} (1972-4) and \textit{Diatope} (1977).  The former was on show in the Cluny Museam in Paris, the latter commissioned for the Georges Pompidou Centre, Paris (of which IRCAM is located at basement level).  Later on, Xenakis used computers to directly synthesise sound, especially from drawing on a special tablet.  This was to be called the UPIC system. (p247).


Also, Gottfried Michael Koenig - in Utrecht.  Other digital systems developed in Utrecht include PILE (Paul Berg, 1975-6), POD  (Barry Truax,  1972-5).  Of these two, POD is the most memorable and Truax survives off it to this very day. The POD system combined both stochastic procedures (algorithms) and direct sound production based around granular synthesis. The POD system (POisson Distribution) developed first at Utrecht, then at Simon Fraser University in Vancouver. In a surprising contrast, later versions resorted back to non real time to allow for more sophisticated control.  Fellow Canadian, Jean Pich\'e also used POD in the late 70s. Early works by Truax include Nautilusin 1976 and Aerial in 1980. Identification with nature through sounds in Pich/'es work \textit{Heliograms} (1978 realised with POD).



\subsection{MIDI}
\begin{itemize}
\item 1982 first-generation computers based on valve technology
\item 1983 second-generation computers based on transistor technology and
\item 1984 third-generation computers based on integrated circuits.
\end{itemize}

Early microcomputers such as IBM, Commodore, Atari, ZX Spectrum etc.

\begin{itemize}
\item First true microprocessor - 4004 Intel Corp.  Then the 8008.
\item By 1975 Motorola had produced the 6800
\item Zilog - Z80
\item MOS Technology - 6502
\item Texas Instruments - variants of the 8008
\item First mini computers - Commadore PET, based on a 6502
\item Radio Shack TRS-80, based on a Z-80 and
\item Apple II based on a 6502.
\item Then Atari - with the 400 and 800 series
\item Early microprocessors were 8 bit.
\item In 1979 Intel produced the 16bit 8086 and Motorrola produced the 68000.
\item In 1981 IBM entered the PC market based on the 8086 - known as the 8088.
\end{itemize}

\subsection{Back to synths}

In 1975 John Appleton, Sydney Alonso, and Cameron Jones working at Dartmouth College, New Hampshire, produced the prototype for an all-digital synthesizer based entirely on a network of integrated circuits and an integral microprocessor.  In 1976 the New England Digital Corporation was established to manufacture the system, known as the Synclavier.


Rival for the Synclavier - Fairlight.
Two Australians, Peter Vogel and Kim Ryrie, made a prototype audio processor called the QASAR M8, completed in 1978 leading in turn to a synthesizer known as the Fairlight Computer Music Instrument or CMI - 1979.  Running from (initially) the 6800 and later the 68000.
German synthesiser - PPG Wave 2.2 was available in late 70s.

Strong projects in Europe centred around IRCAM.
Between 1976 and 1978, Giuseppi di Giugno developed the 4A, 4B 4C and 4X. The 4A consisted of 256 digital oscillators and envelope-generators providing additive synthesis facilities under the control of a PDP 11 computer with up to four different wave-shapes in use at any one time. 

For more details of 4C to 4X (see page 264).  Notable compositions included \textit{Anthony} (1977) by David Wessel using the 4A, \textit{Light} (1979) by Tod Machover using both the 4A and 4C systems.  Machover continued with \textit{Soft Morning, City!}  For tape soprano and double bass. 

WIMP (Windows, Icons, Mouse, Pointer).

Page 275 - establishment of an industry standard protocol to communicate between synths.

Around 1982 meeting of NAMM in January - with reps from CBS/Rhodes, EMU, Kawai, Music Technology Inc. (Crumar) , Oberheim, Octave Plateau, passport Designs, Roland,  Sequential Circuits, Syntauri, Unichord (korg) and Yamaha.

1985- MIDI. (Page 289) - Midi Synthesizers.

\begin{itemize}
\item 1981 - Casio - mini all digital synth VL-1.
\item GS1 GS2 from Yamaha - leading to DX synths and patenting of FM algorithms.
\item 1983 - DX7.
\end{itemize}

MIDI – system exclusive – transfer of samples between samplers and computers say.

MIDI bytes – first bit = status bit (whether the byte is a status byte or data byte).

\subsection{Sampling}

1981 – launch by E-Mu of the Emulator 1. In 1971, Dave Rossum was a grad student at the University of California at Sata Cruz.  From working out the Moog synth, he built his own calling it the E-mu 25. 1981 - E-Mu presented the emulator 1 priced at 10,000 dollars.
Even though the Fairlight (25,000 dollars in 1981) could sample sounds it was very expensive machine.  Emulator has a Z-80 chip inside. 8 bit A-D, D-A, sampling rate at 30kHz and 120kilobytes memory – 4 seconds of sound at maximum fidelity!!!   Resampling algorithm to change pitch (and duration) leaving global sample rate the same so that sounds could be mixed digitally and only one D-A would be required.  Storage on floppy drives, firstly 5.25 inch then the smaller ones we know today.  1.4Meg stores 8 seconds of audio data.  Hard drives 40 – 80 meg.

Through Peter Gabriel in London, and Stephen Pyane, the company Syco was formed.

\begin{itemize}
\item 1984 – Kurzweil 250.
\item 1984 – Emulator II.
\item 1986– Akai S612. 
\item 1987– Casio FZ1
\end{itemize}

\subsection{Japan}
1970s, companies like Roland were formed (Roland founder: Ikutaro Kakehashi). 1960s - Ikutaro was building organs.  Working alongside Hammond. Then formed Roland. 1973 - SH1000. 

There was a gradual need for people to share data and for there to be a common standard. Parallel interfaces (Tom Oberheim's option) was expensive: it needed to be aimed at the consumer.  October 1981 at the NAMM meeting between four Japanese companies - Roland, Yamaha, Korg and kawai, Oberheim and Sequential Circuits (Dave Smith).  Even after a bigger meeting in 1982, only the Japanese were keen so Sequential Circuits began working with them.  First ideas = USI (universal synthesizer interface), Roland came up with Universal Musical Instrument Interface UMII (you-me).  Sequential came back with Musical Instrumental Digital Interface (MIDI). 1983 saw the first machines with it implemented. Version 1.0 was formed between the 5 companies and brought out in August 1983.   Solidarity amongst the Japanese manufacturers led to Yamaha bringing out the DX7 in 1983 at around 2,000 dollars.

1989 the idea of incorporating many samples into one machine - the 'band in a box' principle was started by Emu with their Proteus range.

Synthesizers had onboard sequencers but many were finding that (with MIDI) a computer based sequencer was more efficient.  

\subsection{miscellaneous}
\begin{itemize}
\item Storage: Small Computer System Interface (SCSI)
\item Casio CZ instruments using phase distortion.
\item Kawai 1986 produce the K3 – resynthsis of sampled sounds in memory (not a sampler) Inspired Yamaha to bring out the SY77
\item Korg brought out an FM synthesizer (attacking the strangulating patents Yamaha had put on the process) in their DS8 synthesizer. 
\item 1989 – SY77 and Ensoniq VFX (of which we have both).
\item Digital drum machines – Simmons and Linn were market leaders.
\item Outboard devices for DSP – Yamaha REV1 in 1984 and REV7 in 1985, then the ubiquitous SPX90, to be followed by the SPX1000. With good control via midi.
\item Digidesign Sound Accelerator cards for MAC and Atari. – coprocessing so as not to over burden the computer. Sound Designer first appeared in 1985.  The late 80s saw feverish development of specific DSP chipsets – Intel leading the game.
\item CDP – history (visit the web site) – running Csound and Cmusic on the Atari ST then the TT (a faster version) with the sound streamer (AD DA converters). 
\end{itemize}

Csound – algorithms to physically model sounds – Karplus-Strong. Kevin Karplus and Alex Strong worked on string modelling synthesis at Stanfor during the late 1970s.  Variable length delay line, energized by noise and incorporating a feedback path via an attenuator, coupled to an adjustable low-pass filter which induces resonances akin to those of a vibrating string. 

FOF (Fonction d’Onde Formantique) modelled the voice.  FOF was developed by Xavier Rodet, Yves Potard, and Jean-Baptiste Barrière at IRCAM for CHANT, a software program that uses 5 FOF generators connected in parallel, one for each of the primary formants generated by the vocal tract.  

Dave Oppenheim later a founder of Opcode systems built a computer system as part of a hybrid system for a friend at the Boston School of Electronic Music.  Oppenheim had Oberheim gear and a DX7. he contacted Apple and found their serial ports would run at the MIDI rate so he built a midi interface and wrote a sequencer program.  He was influenced by the PLAY program of Chadabe.  New MIDI software developers emerged: Dr. T, Mark of the Unicorn, Digidesign. David Zicarelli wrote a DX7 editor (using MIDI system exclusive).  Very much Macintosh based.

Dr. T's Music Software began in 1984. Emile Tobenfeld. In 1982, Karl Stienberg was working as a sound engineer at Delta Studio near Hamburg.  He met Manfred R\"urup.  They formed a company, Stienberg in 1983.  Because KS was not a pro programmer he put all the information  on one page, which was new and original. 1986 SR produced the Pro-24 for the new Atari 520ST (with midi interface on board).  Dire Straights were their first endorser. In 1989, Cubase was released on the Atari and Macintosh with a PC version in 1992.  

Such was the speed at which technology developed, people could not only edit MIDI on their home computers, but audio. In 1983, Peter Gotcher and Evan Brooks were in a band together. Together they formed Digidesign initially selling drum chips (ROMMS) for Linn, Oberhiem, Simmons and Emu. In 1985, when the Macintosh became available, they developed Sound Designer, the first commercial digital audio editor.  ``we saw the convergence of a lot of technologies, large hard disks, digital to analog converters, Macs with slots''.

Not only in the commercial world.

 IRCAM had its mac studio and Yamaha studio.  Adrian Freed started the MIDI LISP project.  Goal, to provide a front end program for someone to write code in lisp (based around Le LISP) to control a MIDI synthesizer. 1984, Jean - Bajptiste Barri\`ere became head of music research at IRCAM.  ``how to solve instrumental writing techniques with computers''.  Hence a series of amanuensis programs developing out of the LISP base.   First exmaples - Esquisse a program written in Le Lisp on the Macintosh Plus. Mikael Laurson took the next step in devising a graphical interface called Patchwork. (as is used today but now morphed into Open Music).   

Commercially in the states, Intelligent music published M and Jam Factory written by David Zicarelli.
   
At Ircam he met Miller Puckette the inventor of MAX and began to work on the commercialisation of the software.  Intelligent music published the program.  He made changes to the scheduler so that the user could still interact with the system whilst it was working.  Page 208 bottom to see the usual financial problems that seem to crop up meaning that in the end, by 1990, Opcode published MAX.  Latest history is that Opcode were (to be) purchased by Gibson (guitars never die - but Gibson is apparently a monster and might be worth looking at) and finally sold MAX (and its synth partner MSP) to Zicarelli's company cycling74.

\subsection{France}
Meanwhile in France – the GRM in 1970s were bringing their studios up to date – with the SYTER (Syst\`eme Temps R\'eel). First complete version finished in 1982 – realtime sound transformation. SYTER became GRM tools around 1989 by Hugues Vinet.



