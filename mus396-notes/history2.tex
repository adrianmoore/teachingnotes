%\chapter{USSS toolkits}
%shown in figure~\ref{fig:artificialnatural}
%pages~\pageref{tab:opposites1}
%and form (section~\ref{section:form}, page~\pageref{section:form}).

\chapter{Electronic music 2}
\label{history2}

\section{Early American Electronic Music, East coast and West coast differences, Cage, Wendy Carlos and others. Modular synthesizers of Moog and Bucla. Music N languages.}
\begin{itemize}
\item Synthesizers
\item John Cage
\item Forbidden Sounds. (Louis and Bebe Barron)
\item The modular synths of Moog and Buchla
\item Switched on Bach - Wendy Carlos
\item Early Music N
\end{itemize}

Bearing in mind the huge leaps made at the radio stations in Europe and in particular the work of Karlheinz Stockhausen, it is surprising just how the Americas could take such a radical alternative route (or indeed routes).

\begin{itemize}
\item 1959-60 \textit{Kontakte} (electronic and version for piano, percussion and electronics)
\item 1966 \textit{Telemusic} (electronic music)
\item 1966 \textit{Hymnen} (electronic and concrete music) - version with soloists - 1969 version with orchestra
WDR, RTF = radio
\end{itemize}

but in America you see RCA (Radio Corp of America) academia + the large foundations (Guggenheim/Rockefeller)

RCA Helped produce and market the Theremin (if only briefly)

In 1939 Cage performed his Imaginary landscape No. 1 in Seattle for muted piano, cymbal and two variable-speed turntables playing RCA Victor test recordings of fixed and variable frequencies. Live electronics preceded tape music by a decade.

Hugh Le Caine, scientist who worked on the development of radar during WWII and afterwards nuclear physics. Best known for the design of an Electronic Sakbut (1945), The National Research Council, Ottawa established a research studio for him. He also designed a multi-track tape recorder (1953). One of his best known works is \textit{Dripsody}(1955) based upon the sound of a drop of water.

RCA started to build synthsizers.

Columbia University in New York City became interested in electronic music. Vladimir Ussachevsky (1911-1990) \textit{wireless fantasy} (1960) – the first American electronic music pioneer and lecturer - held concerts of electronic music. Ussachevsky and Otto Leuning (1900-1996) low-speed (1952) moved their studio from Ussachevsky’s living room into University accommodation. RCA became involved around the same time as Milton Babbitt showed interest at Princeton. In January 1959 the Columbia – Princeton Electronic Music Center (containing a Mark II RCA synthesizer) was established. (for picture see Manning)

Early Cage - landscape

USA: Louis and Bebe Barron founded a personal studio. Working as early as 1948 with taped sounds. In 1956 they made the film score for Forbidden Planet. They worked with John Cage (1912 -1992) in 1951 where the Project for Music for Magnetic Tape was initiated. They worked on the first Williams Mix recording 600 sounds. The cutting and splicing for Williams Mix was finished in late 1952.
\begin{enumerate}[a]
\item City sounds
\item Country sounds
\item Electronic sounds
\item Manually produced sounds including normal music
\item Wind produced sounds including voice
\item Small sounds requiring amplification to be heard.
\end{enumerate}

C indicated control and predictability, V described lack of control or unpredictability

C and V applied to pitch, timbre and loudness. Therefore \textbf{B v c v} implies country uncontrolled pitch, known timbre, uncontrolled loudness.


In 1960 Mario Davidovsky arrived in New York from Argentina. The first concert included in 1961 Davidovsky's \textit{electronic study number 1}, and works by Ussachevsky, Babbitt, Luening, Charles Wuorinen. The Duo gave a performance of their Tape Music on 28 October 1952. Berio was in the audience.

Others working with Cage on `music for magnetic tape' were Feldman, Brown, Tudor and Wolff. Technically, similar processes to that of musique concr\`ete and elektronische Musik but aesthetically bound around Cage's ideas of indeterminacy. The project terminated in 1953. Brown and Feldman worked together for a while (producing Octet1 - for eight loudspeakers). After Cage had made his visit to Milan in 1959 he worked consistently with David Tudor and began to use electronics in live performance. Fontana Mix was more dramatic than Williams Mix and was linked with Aria - a live vocal part added to the tape in performance.

(Still in the USA), Var\`se had struggled to gain a footing for many years. After Density 21.5 (1936) there were no works produced for 18 years. Thus in the late 40s Var\`ese began formulating plans for a work for ensemble and tape (D\'eserts). He had completed the instrumental parts by end 1952. In 1953 he recorded iron mills, saw mills and other factory noises in Philadelphia and used them back at his house in Greenwhich, New York. Boulez visited America in 1953 to present a concert of musique concr\`ete and met Var\`ese. In 1954 Var\`ese received an invitation to go to Paris from Schaeffer to complete the tape parts. His liberal approach to organising sound was a shock to Schaeffer and Var\`ese worked very quickly, completing the tape part in two months. The results weren't great however. The first performance took place in the Th\'e\^atre des Champs-Elysées on 2 December 1954 conducted by Hermann Scherchen. Forty years previously, this hall had seen riots at the premiere of Stravinsky's Rite of Spring. Similar noises were made after D\'eserts. Var\`ese returned to the US in 1955 and D\'eserts was performed in 30 November 1955 in New York.

June 1955 Ussachevsky and Luening received 10,000 dollars from the Rockefeller foundation funded through Barnard College to investigate the state of studio facilities at home and abroad. Ussachevsky and Luening discovered some research taking place at Bell Telephone laboratories, New Jersey into computer analysis and (re)synthesis of sounds. The duo made a successful application to create a studio at Columbia University. Whilst Ussachevsky and Luening wanted grant monies to go to Universities where research could take place without commercial pressures, RCA jumped ahead of the game a little by producing their first self-contained sound synthesizer on 31 January 1956.

\subsection{Italy}
Berio was at a concert of electronic music on October 28, 1952.  Made friends with Ussachevsky and Luening.  Back in Milan in 1953 he met with Bruno Maderna (who had made \textit{Musica su Due Dimensioni} in 1952). Thus was born the in 1955 the Studio di Fonologia Musicale at RAI (Radio Audizioni Italianne) in Milan.  Works to try to get hold of include \textit{Ritratto di Citt\'a} a sound portrait of Milan during the course of a day, with a text by Roberto Leydi – radiophonics.

The Milan studio was not duty bound to a Paris or Cologne aesthetic.  Berio's first important piece at the Milan studios was \textit{Thema – Omaggio a Joyce} (1958).  Working with Umberto Eco looking at Joyce's Ulysses. The text (see page 49 of Chadabe) was recited by Cathy Berberian (later Mrs Berio). There are scales of speech/onomatopoeia fused with musical intention.

Cage present in Italy for a realization of \textit{Fontana Mix} in 1958.  Berio's last tape work was \textit{Visage} (1961) which is a radio work – singing against an ambient electronic track.  Berio left for the US, but Maderna stayed behind, completing \textit{Continuo} (1958), \textit{invenzione su una Voce} (1960), and \textit{Serenata III} (1961).  But Maderna was very busy with conducting engagements so the principal composer became Luigi Nono. Best known for his political work – \textit{La Fabbrica Illumanata} (1964, The illuminated Factory) spotlighting the plight of factory works. (sounds from a steel plant in Genova).

\subsection{UK}
Tristram Cary's studio was an independent. He purchased a small amount of equipment and began some small jobs for the BBC as an independent.  Music for films such as The Lady Killers, (1955), The Little Island (1958) made on zero budget.  Cary was a prolific composer.  More work for the BBC (radio plays), and finally Doctor Who. 

1950 – 1960 saw many studios begin (and some fail). The Radiophonic Workshop opened on April 1, 1958 – headed by Brian Hodgson
Dr. Who in 1963 – Tristram Cary and Ron Grainer (who wrote the theme music) with the help of Delia Derbyshire. \url{http://www.deliaderbyshire.com/}

\subsection{Sweden}
Music Text works (again radio inspiration behind most of it) from Lars-Gunnar Bodin and Bengt Emil Johnson.  In 1966, Bodin, Johnson, Sten Hanson and others formed Fylkingen.  Text-sound composition focusing upon the sounds of works as well as meanings.  

EMS – the national studios of Sweden (Elektronimusikstudion) were completed in 1972.  Composers of note from the early part of EMS's history are Lars Gunner Bodin, Knut Wiggen and Tamas Ungvary (see web site). Later star composers include Ake Parmerud and Erik-Michael Karlsson. 


\section{Synthesizers}

In 1949 Robert Moog read an article on how to build Theremins. Moog set up his company after working with Herb Deutsch at Hofstra University. Moog learnt from an article written by Harald Bode. Moog then developed more kit upon orders from the likes of Ussachevsky and Hiller. And Wendy Carlos.

\href{http://www.wendycarlos.com/}{Wendy} (Walter became Wendy in 1972) Carlos who went on to play the original music to \textit{A Clockwork Orange}.

Best known for \textit{Switched on Bach} – 1968 (Examples - Disc 1 (track 6, track 14), Disc 4 (extended)).
Carlos who went on to play the original music to A Clockwork Orange. 

concept of the modular synthesizer.  The word synthesizer used by Moog first in 1967.

1963 San Francisco Tape Music Center – arrival of Donald Buchla.  The ideas of working with tape (cutting and splicing) were being superseded by synthesizers  and the like. Buchla designed something for Subotnick and Sender.  Buchla included an idea for a sequencer.  Stored voltages that automated events.  For more information on how it works see – manning.  Subotnick, Buchla made an application to the Rockerfeller foundation for 500dollars and a 16 step sequencer with touch sensitive keyboard was built. Their work together was to produce the series 100 which eventually Buchla sold the rights to CBS.   Subotnick received commissions from Nonesuch Records – for \textit{Silver Apples of the Moon} (1967), \textit{The Wild Bull} (1968) and \textit{Touch} (1969). All completed on the Buchla.

It took the likes of Wendy Carlos' Switched On Bach and Keith Emerson of Emerson, Lake and Palmer with \textit{lucky man} to bring success to the minimoog – it became a fashion accessory for the pop world.

\subsection{In London}
1966 Peter Zinovieff in London also found the need to eliminate tape.  He built a sequencer and worked from home.  Influenced by Moog naturally. Zinovieff and Cary collaborated and moved a computer in to control the sequencing. (Chadabe 150 for the anecdote). They booked the QEH for a date in January 1968. To make ends meet Zinovieff formed EMS Ltd.  From a bequest they made the cheap VCS-1, the expansion of which became the VCS-3.  Then came the Synthi A and the Synthi 100 \url{http://www.ems-synthi.demon.co.uk/} - gives you an idea of current retail value (1800 for a VCS3).

The EMS firm employed over 40 people at that time. At their studio composers such as Cary, Harvey and Hans Werner Henze worked for a time. HWH made \textit{Glass Music} (1970), \textit{Prison Song} (1971) and \textit{Tristan} (1973). Also Birtwistle.  His works included \textit{Four Interludes} (1969), \textit{Medusa} (1970), \textit{Signals} (1970), \textit{Chronometer} (1971) \textit{Orpheus} (1976). 

%%%%%%%%%%%%%%%ballora

Moog continued to market his modular systems after his first orders at the AES convention of 1964. With transistors enabling small, portable devices, he was able to sell component systems consisting of oscillators, amplifiers, noise generators, and filters. The components could be interconnected by patch cables to create a variety of sounds limited primarily by the number of available modules and the synthesist's imagination. In 1967, he began calling it a synthesizer, apparently realizing it had much in common with the RCA synthesizer at Columbia-Princeton. The voltage controlled modular synthesizer was based in subtractive synthesis, which involves the use of complex waves or noise and filters.

Don Buchla: By 1965, the composers at the San Francisco Tape Music Center were tiring of the endless splicing work they were having to do to produce tapes. So Don Buchla created an automation system with voltage controlled modular components. It featured a sequencer, which could store a series of voltage values. These could then be applied to a number of musical parameters, such as pitches, volume levels, filter cutoff frequencies. It was in some ways an electronic player piano.

Besides the feature of the sequencer, another principal difference between Buchla's and Moog's instrument was that Buchla's instrument was not based on a piano keyboard. His instruments featured a series of capacitance plates. Pressing on them changed the capacitance of the plates, and their changing voltage levels could be directed to different musical parameters.

In 1966, following an order of three of Buchla's systems by Vladimir Ussachevsky, Buchla founded the company Buchla and Associates, which marketed the Modular Electronic Music System. Buchla did not care for the term "synthesizer," largely due to the hyperbolic claims made by RCA about its instrument in the 1950s ("capable of producing any sound in nature"). Rather than producing a universal sound producing machine, he was interested in creating novel musician's interfaces, which he simply called "Electronic Music Boxes." Principal users of the Buchla synthesizer were David Tudor of Merce Cunningham's dance company, and composer Morton Subotnick of the San Francisco Tape Music Center. While the Moog was to become the more commercially successful, the Buchla remained more of an elite fringe instrument that influenced many innovators of the time. Principal users of the Buchla synthesizer were David Tudor of Merce Cunningham's dance company, and composer Morton Subotnick of the San Francisco Tape Music Center. The Buchla Box was a featured component in many of the mixed media performances emerging from the Haight-Ashbury psychadelic movement. (The Tape Music Center, being in the same neighborhood, was a natural participant in these events, but the Center officially separated itself from them, concerned that the drug associations of these events might hurt the long-term funding that the Center was pursuing.)

Synthesizer Recordings: The team of Paul Beaver and Bernie Krause brought the "Moog sound" to the Los Angeles societies of popular music and Hollywood. Their work is heard on a number of signature early synthesizer recordings. Among them are the album Cosmic Sounds an assemblage of music arranged by Mort Garson, poetry, and electronic effects. It remains a hallmark of the psychadelic era, right down to its liner instructions: ``MUST BE PLAYED IN THE DARK.''  This, plus the Byrds' The Notorious Byrd Brothers, put them in the right place at the right time. They became the West Coast representatives of the Moog company at a time when psychadelic rock music was beginning to look like big business to the record companies. Following these recordings, they set up a performance/demo at the Monterey Pop Festival of 1967. This was the first meeting of a number of rock streams, from San Francisco hippies to rockers from New York and London, many of them running around with large advances they'd recently received from record companies. Once they heard they Moog, they all placed orders. The psychedelic era represented a culture of breaking down boundaries and new transformations. The synthesizer, with its ability to create new sounds and shape them in ways never heard before, was an ideal tool not just for music, but for the ideal of reshaping consciousness

Few of these artists, however, cared to learn to use the instrument, so they Beaver and Krause found themselves in demand as synthesizer engineers/consultants/sound designers for a number of artists including the Doors, the Beatles, the Beach Boys, Frank Zappa, the Monkees, and Crosby, Stills, Nash and Young. They also did a numerous effects for films and commercials. Their iconoclastic spirit was perhaps best shown in their production of an ad for a company named ITT, whose logo was a radio tower and beacon. Irritated at the company's unwillingness to pay going rates for their work, they made a design based on a Morse code spelling out the words ``F*** You'' and delighted in the fact that no one ever noticed. They also produced a classic album and technical manual, The Nonesuch Guide to Electronic Music, which, in the synthesizer craze of the time, actually became a hit. (It can be found in the PSU Library.)

The first all-synthesizer album was created by Morton Subotnick (1933-) with the Buchla synthesizer. Subotnick had been one of the composers at the San Francisco Tape Music Center, who was now teaching at New York University and had brought his Buchla with him. Called Silver Apples of the Moon, it represented a rather bold step by Nonesuch Records. It sold in classical music sections, and was quite popular when it came out. Both Subotnick and Nonesuch were taken with the idea of creating music that was to be played in people's living rooms over their stereo systems. The album's release signified the ubiquitous presence that audio equipment had assumed in the American home. The title was from a poem by W.B. Yeats ("The Song of Wandering Aengus"), and was meant to transport listeners to a variety of sonic landscapes, sometimes with no transition. It was characterized by prominent use of the Buchla sequencer to create filigrees of notes.

Subnotnick followed this release up with The Wild Bull, released in 1968. This piece was based on a Sumerian poem, narrated by a woman who was lamenting the death of her husband in a war. It began with the text ``The wild bull who was laying down...'' Subotnick attempted to create a piece about war and loss. This piece was also characterized by the sequencer, but it is a bit more understated. Some of the effects are obtained by applying the sequencer to filter cutoff frequencies, rather than notes, so that timbral changes were looped. The piece also has as a prominent element a descending low tone, meant to suggest a moan of lament. It has a lower, heavier sound overall, to suggest a sense of antiquity and loss, the timelessness of mourning early deaths.

The breakthrough came with Switched-On Bach, by Wendy Carlos (1939-), in 1968. Carlos had done graduate work at Columbia with Ussachevsky and Luening, worked as a recording engineer after graduating, and had become friendly with Bob Moog as she explored and helped developed the modular synthesizer. It was the result of countless hours of painstaking studio work. Synthesizers at the time were monophonic. For legato passages, where notes overlapped, the parts had to be recorded one note at a time, on separate tracks of a multitrack tape deck. The pieces had to be thought through before they were recorded. To give them life, a click track would have to be recorded that incorporated slight changes in tempo. This was achieved with a single pitch being controlled by a low frequency oscillator. The frequency of the LFO was controlled by the keyboard, with lower notes to slowing it down and higher notes speeding it up. She recalled in a 1979 interview that the difficulty in working with the click track was first learning to play to it, and then learning to take that a step further and not play right on the beat, but "around" the beat. Frequently, each successive note in a passage had a different envelope assigned to it, such as alternating between quick, "percussive" attacks and slow attacks, in emulation of a string player changing bowing directions for each note. Thus, the album featured an obsessive attention to detail down to every note, timbre, articulation and rhythm. She was later to describe the process as being analogous to the way early Disney cartoons were created, with one frame drawn at a time. The efforts were not in vain. The album won a Grammy award and became the first classical recording ever to go platinum. It brought the synthesizer into popular consciousness, with features appearing in The New York Times and Newsweek. Carlos became the superstar of the electronic music movement.


Carlos followed this up with other releases of synthesized classical music, while gaining further exposure with the soundtrack to Stanley Kubrick's film A Clockwork Orange, in which the electronic musical score reflected the disturbing futuristic setting of the film. The soundtrack featured both realizations of classical works and original compositions. Among the disturbing elements for many viewers was the use of the vocoder for singing synthesizer effects. This was the first time the device had been used in a music recording, and many listeners did not know what to make of it. The synthesizer became the musical flavor of the month. Dozens of ``Switched-On'' albums were released consisting of tongue-in-cheek renditions of music in all genres. Many of these albums are now collector's items of period kitsch, notable probably more for their cover art than for the music they contain. Sales came nowhere near those of the Carlos albums. Shortly after the release of Switched on Bach, the band Emerson, Lake and Palmer released the song ``Lucky Man.'' It featured a synthesizer solo by Keith Emerson, played on a Moog that knocked the socks off rock and roll listeners. Synthesizers were no longer arcane technical puzzles, but had reached the level of proletarian musicians. Now any self-respecting keyboardist had to have one.

A number of composers began to explore the use of synthesizers and sequencers as composition/improvisation tools. Larry Austin used a Buchla synthesizer and controlled its sequencers with the touch pads, so that by interacting with the machine he could control the rate of sequencing. Some sequencers triggered other sequencers, so that all of the processes could interact with each other. The prize for largest collection of Moog sequencers went to Joel Chadabe (1938-), who worked with Bob Moog to create an automated synthesizer system called CEMS (Coordinated electronic Music Studio) from 1967-1969. It had a large array of sound generating modules, as well as a digital clock and eight sequencers that could be run synchronously, asynchronously, one after the other, or in any combination. Chadabe used the system to create processes with varying degrees of determinacy and randomness so that compositions could be realized as the system played.

The fortunes of the Moog company had taken a downnturn after the initial post Switched-On Bach rush. At the same time, engineers and consultants were finding that they used the same patch configurations time and time again. The company began to design a smaller instrument, more suitable for mass production In 1969 the MiniMoog was released.  This was a smaller, more portable synthesizer on which connections between modules were made at the push of a button, rather than via patch cables. While not as flexible as the large modular systems, it could be sold in retail stores and was practical for stage use. With its handsome walnut case and flip-up control panel, it was much more an instrument than an electronic monstrosity. As the first affordable and portable synthesizer, it was featured on many rock recordings in the 1970s. It was largely responsible for ``democratizing'' the synthesizer. The patching was all internal, consisting of a few basic configurations, eliminating the "spaghetti factor." It featured two expression controllers, the pitch bend wheel and the modulation wheel, which controlled vibrato. These two expression wheels became standard features on synthesizer instruments (and remain so to this day).

In 1969 the ARP synthesizer company was launched, and the synthesizer industry was born with competing companies and products competing for the highest "cool" rating. In London the VCS synthesizer was used by artists such as Pink Floyd and Brian Eno, and was sold briefly in the United States as the Putney. Along with the industry came a new focus in keyboard circles not just of traditional playing techniques, but attention to timbre. Gear magazines such as Keyboard and Electronic Musician had appeared by the end of the 1970s.
Perhaps the most unusual and difficult to maintain instrument was the Mellotron, which offered an alternative to synthesizers by featuring lengths of recorded tape of different instruments. Each length of tape lasted approximately eight seconds, and each was assigned to a different key, thus producing a recording of an instrument at the pitch of the depressed key. The tapes did not loop, so the duration per note was limited to the length of tape. Mellotron tapes were often three-track, enabling sounds to be changed on the fly via a switch that shifted the playback heads. This was popular for a number of years with rock bands that could afford to maintain and/or transport the instrument. Perhaps the most famous recordings are the flutes that are featured in the introductions of the Beatles' ``Strawberry Fields Forever'' and Led Zeppelin's ``Stairway to Heaven.''






