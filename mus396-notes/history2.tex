%\chapter{USSS toolkits}
%shown in figure~\ref{fig:artificialnatural}
%pages~\pageref{tab:opposites1}
%and form (section~\ref{section:form}, page~\pageref{section:form}).

\chapter{Electronic music 2}
\label{history2}

\section{Synthesizers 1}
\begin{itemize}
\item Stockhausen recapitulated.
\item Synthesizers
\item John Cage
\item Forbidden Sounds. (Louis and Bebe Barron)
\item The modular synths of Moog and Buchla
\item Switched on Bach - Wendy Carlos
\end{itemize}

\begin{itemize}
\item 1959-60 \textit{Kontakte} (electronic and version for piano, percussion and electronics)
\item 1966 \textit{Telemusic} (electronic music)
\item 1966 \textit{Hymnen} (electronic and concrete music) - version with soloists - 1969 version with orchestra
WDR, RTF = radio
\end{itemize}

but in America you see RCA (Radio Corp of America) academia + the large foundations (guggenheim/rockefeller)

RCA Helped produce and market the Theremin (if only briefly)

In 1939 Cage performed his Imaginary landscape No. 1 in Seattle for muted piano, cymbal and two variable-speed turntables playing RCA Victor test recordings of fixed and variable frequencies. Live electronics preceded tape music by a decade.

Hugh Le Caine, scientist who worked on the development of radar during WWII and afterwards nuclear physics. Best known for the design of an Electronic Sakbut (1945), The National Research Council, Ottawa established a research studio for him. He also designed a multi-track tape recorder (1953). One of his best known works is \textit{Dripsody}(1955) based upon the sound of a drop of water.

RCA started to build synthsizers.

Columbia University in New York City became interested in electronic music. Vladimir Ussachevsky (1911-1990) \textit{wireless fantasy} (1960) – the first American electronic music pioneer and lecturer - held concerts of electronic music. Ussachevsky and Otto Leuning (1900-1996) low-speed (1952) moved their studio from Ussachevsky’s living room into University accommodation. RCA became involved around the same time as Milton Babbitt showed interest at Princeton. In January 1959 the Columbia – Princeton Electronic Music Center (containing a Mark II RCA synthesizer) was established. (for picture see Manning)

Early Cage - landscape

USA: Louis and Bebe Barron founded a personal studio. Working as early as 1948 with taped sounds. In 1956 they made the film score for Forbidden Planet. They worked with John Cage (1912 -1992) in 1951 where the Project for Music for Magnetic Tape was initiated. They worked on the first Williams Mix recording 600 sounds. The cutting and splicing for Williams Mix was finished in late 1952.
\begin{enumerate}[a]
\item City sounds
\item Country sounds
\item Electronic sounds
\item Manually produced sounds including normal music
\item Wind produced sounds including voice
\item Small sounds requiring amplification to be heard.
\end{enumerate}

C indicated control and predictability, V described lack of control or unpredictability

C and V applied to pitch, timbre and loudness. Therefore \textbf{B v c v} implies country uncontrolled pitch, known timbre, uncontrolled loudness.


In 1960 Mario Davidovsky arrived in New York from Argentina. The first concert included in 1961 Davidovsky's \textit{electronic study number 1}, and works by Ussachevsky, Babbitt, Luening, Charles Wuorinen. The Duo gave a performance of their Tape Music on 28 October 1952. Berio was in the audience.

Others working with Cage on `music for magnetic tape' were Feldman, Brown, Tudor and Wolff. Technically, similar processes to that of musique concr\`ete and elektronische Musik but aesthetically bound around Cage's ideas of indeterminacy. The project terminated in 1953. Brown and Feldman worked together for a while (producing Octet1 - for eight loudspeakers). After Cage had made his visit to Milan in 1959 he worked consistently with David Tudor and began to use electronics in live performance. Fontana Mix was more dramatic than Williams Mix and was linked with Aria - a live vocal part added to the tape in performance.

(Still in the USA), Var\`se had struggled to gain a footing for many years. After Density 21.5 (1936) there were no works produced for 18 years. Thus in the late 40s Var\`ese began formulating plans for a work for ensemble and tape (D\'eserts). He had completed the instrumental parts by end 1952. In 1953 he recorded iron mills, saw mills and other factory noises in Philadelphia and used them back at his house in Greenwhich, New York. Boulez visited America in 1953 to present a concert of musique concr\`ete and met Var\`ese. In 1954 Var\`ese received an invitation to go to Paris from Schaeffer to complete the tape parts. His liberal approach to organising sound was a shock to Schaeffer and Var\`ese worked very quickly, completing the tape part in two months. The results weren't great however. The first performance took place in the Th\'e\^atre des Champs-Elysées on 2 December 1954 conducted by Hermann Scherchen. Forty years previously, this hall had seen riots at the premiere of Stravinsky's Rite of Spring. Similar noises were made after D\'eserts. Var\`ese returned to the US in 1955 and D\'eserts was performed in 30 November 1955 in New York.

June 1955 Ussachevsky and Luening received 10,000 dollars from the Rockefeller foundation funded through Barnard College to investigate the state of studio facilities at home and abroad. Ussachevsky and Luening discovered some research taking place at Bell Telephone laboratories, New Jersey into computer analysis and (re)synthesis of sounds. The duo made a successful application to create a studio at Columbia University. Whilst Ussachevsky and Luening wanted grant monies to go to Universities where research could take place without commercial pressures, RCA jumped ahead of the game a little by producing their first self-contained sound synthesizer on 31 January 1956.

\section{Synthesizers 2}

In 1949 Robert Moog read an article on how to build Theremins. Moog set up his company after working with Herb Deutsch at Hofstra University. Moog learnt from an article written by Harald Bode. Moog then developed more kit upon orders from the likes of Ussachevsky and Hiller. And Wendy Carlos.

\href{http://www.wendycarlos.com/}{Wendy} (Walter became Wendy in 1972) Carlos who went on to play the original music to \textit{A Clockwork Orange}.

Best known for \textit{Switched on Bach} – 1968 (Examples - Disc 1 (track 6, track 14), Disc 4 (extended)).

1963 San Francisco Tape Music Center – arrival of Donald Buchla. The ideas of working with tape (cutting and splicing) were being superseded by synthesizers and the like. Buchla designed something for Morton Subotnick and Ramon Sender. Buchla included an idea for a sequencer. At the request of Subotnick, Buchla made an application to the Rockerfeller foundation for 500 dollars and a 16 step sequencer with touch sensitive keyboard was built. Their work together was to produce the series 100 which eventually Buchla sold the rights to CBS. Subotnick received commissions from Nonesuch Records – for \textit{Silver Apples of the Moon} (1967), \textit{The Wild Bull} (1968) and \textit{Touch} (1969). All completed on the Buchla.

Example (silver apples of the moon). A mechanical landscape: Sequencers, gates, sample and hold circuits. Energy. Part II highlights repeating gestures. Sequencers of 8 or 16 degrees were generated by Buchla. The title of the work comes from a poem by Yeats.

Two other rivals appeared later: Tonus, / ARP in America and EMS Ltd., pinoeered by Peter Zinovieff in England. Voltage control enabled the development of the keyboard to control oscilators. Oscillators could control oscillators. Output of an oscillator is connected across the frequency control break-point of a second … vibrato. The speed is controlled by the frequency of the oscillator, the depth by the amplitude. If the depth of modulation is pronounced and the freq is high (order of 12 hz) it becomes impossible to track the frequency of the wave. (FM)

