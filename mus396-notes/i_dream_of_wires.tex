\chapter{I dream of wires}.

A film by Robert Fantinatto and Jason Amm

\section{part 1: Early history}
Non-computer. Non-internet
Modular synthesizer. 

Coming from the Theremin. Labs and inventors. 
Electronic Sackbutt - Canadian Hugh LeCaine. 1940s.
Repurposed lab equipment and radio test generators. 

Music departments - electronic music studios in Europe and North America. 
Joel Chadabe. EMF. Was working at SUNY Albany. 

Morton Subotnick. PVS station. Score for 'Computer in the mind of man'
So he brought and oscillator (sine/square wave) 

Columbia University Electronic Music Studio in the old venue of the Manhattan Project.
The RCA Mark II synthesizer. Vacuum tubes. 120,000 dollars then. 

(Terry Pender and Brad Garton working today and on the show)

Then the transistor took over. 

Undergraduate in Physics at Columbia - Robert Moog. Moog was selling DIY Theremins. Small store in up-state New York.
Herb Deutsch wanted an oscillator with precise control. Moog added Voltage control (1v per octave). 
Connecting these things together. Working with Universities. Then voltage controlled filters. 
Moog 904 series filter. Mid-1960, brand new sounds. 

East Coast synthesizer design. 

On the West Coast. Hippies and counter culture. 

Mills College Berkley San Fransisco Tape Music Center. 

Buchla synth. Don Buchla - former NASA engineer. 
100series modular electronic music system. 

Ramon Sender and Morton Subotnick worked with Buchla. Buchla voltage control scheme was not 1v per octave so it made more extreme soundworlds. So the West coast philosophy was much more experimental as it wasn't so tied to the rules. East cost was more pitch based and lots of 4-pole filter.  

Moog - audio could also be control but you could put a keyboard on it.
Buchla - required multiple wires but you could stack them. He didn't put a keyboard on though he used a touch sensitive pad. He also created the sequencer. 

Monteray pops festival. Bernie Krause introducing the Moog 16,000 dollars each.
The nonesuch guide to electronic music.  

So Subotnick moved to New York as the San Fransisco scene was too `free'. Silver Apples of the Moon. But completely overtaken by Walter now Wendy Carlos' \textit{Switched on Bach}. 

Then the sequencer moved into popular music.

ARP2500, EMS Synthi 100, Roland System 100M and System 700. Look at the price!!!! 
EMS - pin and matrix system and in the States EML. 
EMS - VCS3

Moog created the \textit{minimoog} in 1970 which gave great portability. Then ARP made the 2600 - connections were pre-wired but could be overridden. Sequencer built in. 

Cheap alternatives Serge Chertnin (again a music professor) - DIY. 700 dollars for parts then build it yourself.   

Punk - Gary Newman was in a band. 

then cheaper synthesizers from Japan flooded the market. Korg 700S and SH101. 
And digital - with presets. 

Academic electronic music centers. Then the computer came along. Hybrid synthesis - digital synthesis with analogue control. 

Synclavier (New England Digital)and Fairlight CMI. 

Analogue killer was the DX7 Yamaha. 

1993 Analogue became `vintage'. (Jack Dangers, Meat Beat Manifesto). EMS - vocoder 2000.   
EMS - Synthi 100. 7200 physical inputs. 

End of the 80s - Acid House. 808, 909 and 303. 

Modulars back into fashion. the group \textit{Node}. Brian Eno and the idea of doing installations. 

\section{part 2: The return of the modular}
Keeping the old synthesizers alive but also re-inventing them. 
Names: Jon Sonnenberg (Travelogue). Flood. Gert Jalass (Moon modular) David Dixon (intellijel) Jeff Blenkinsopp (The Analogue Lab)repairs old synthesizers. 

Solvent (aka Jason Amm) electronic dance music.  

Bruce Duncan - interested in the hardware. Modcan synthesizer.

Dieter Doepfer. The latest in modular. Eurorack.

Buchla 200E

NAMM show Anaheim California.

