%\chapter{USSS toolkits}
%shown in figure~\ref{fig:artificialnatural}
%pages~\pageref{tab:opposites1}
%and form (section~\ref{section:form}, page~\pageref{section:form}).

\chapter{Electronic music 8}
\label{history8}

\section{Mixed Music and Interaction, Boulez, Manoury, IRCAM, STEIM, ZKM.}

In Italy in the mid-1960, Giuseppe Di Giugno was teaching physics at the University of Naples.  Got in to Electronic music.  Meeting with Di Giugno and Berio for a new system at IRCAM.  During 1975 Di Giugno built the 4A.  Di Giugno met Hal Alles (another synth builder) to make the 4B, then the 4C (more programmable).  Then finally in 1980 the 4X which was very transparent and programmable: first early performance of Boulez's \textit{R\'epons}.  \textit{R\'epons} was extremely elaborate.  Others utilised it in a less significant way: Luciano Berio, Philippe Manoury, Pierre Henry, François Bayle, Denis Lorrain, George Benjamin, Marco Stroppa.  Marc Battier and Cort Lippe were IRCAM tutors. SEE BORN here for gossip. The 4X cost 100,000 dollars.  Therefore they had to think about microelectronics. The 4X became MAX/MSP 

Di Giugno met Paolo Bontempi in Rome who had just purchased Farfisa.  In 1985 Miller Puckette arrived at IRCAM and worked on 4X related software with Philippe Manoury.  Manoury's works included \textit{Jupiter} (1987) that accepted realtime flute sounds and triggers. Puckette needed a realtime scheduler. ``I used ideas that I had learned from Max Mathews and so I ended up calling the program MAX.''

At first, the Macintosh provided the MIDI and hooked into the 4X.  Then the interface became increasingly more significant. 




