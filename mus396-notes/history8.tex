%\chapter{USSS toolkits}
%shown in figure~\ref{fig:artificialnatural}
%pages~\pageref{tab:opposites1}
%and form (section~\ref{section:form}, page~\pageref{section:form}).

\chapter{Electronic music 8}
\label{history8}

\section{Mixed Music and Interaction, Boulez, Manoury, IRCAM, STEIM, ZKM.}

In Italy in the mid-1960, Giuseppe Di Giugno was teaching physics at the University of Naples.  Got in to Electronic music.  Meeting with Di Giugno and Berio for a new system at IRCAM.  During 1975 Di Giugno built the 4A.  Di Giugno met Hal Alles (another synth builder) to make the 4B, then the 4C (more programmable).  Then finally in 1980 the 4X which was very transparent and programmable: first early performance of Boulez's \textit{R\'epons}.  \textit{R\'epons} was extremely elaborate.  Others utilised it in a less significant way: Luciano Berio, Philippe Manoury, Pierre Henry, François Bayle, Denis Lorrain, George Benjamin, Marco Stroppa.  Marc Battier and Cort Lippe were IRCAM tutors. SEE BORN here for gossip. The 4X cost 100,000 dollars.  Therefore they had to think about microelectronics. The 4X became MAX/MSP 

Di Giugno met Paolo Bontempi in Rome who had just purchased Farfisa.  In 1985 Miller Puckette arrived at IRCAM and worked on 4X related software with Philippe Manoury.  Manoury's works included \textit{Jupiter} (1987) that accepted realtime flute sounds and triggers. Puckette needed a realtime scheduler. ``I used ideas that I had learned from Max Mathews and so I ended up calling the program MAX.''

At first, the Macintosh provided the MIDI and hooked into the 4X.  Then the interface became increasingly more significant. 

Xenakis - designs for global pictures for \textit{Metastatis} and \textit{Pithoprakta}.  Using the computer for global outlines perhaps. In 1972 he formed the CEMAMu (Centre d'Etudes de math\`ematiques et Automatiques Musicales). Here he developed UPIC (Unit\'e Polyagogique Informatique du CEMAMu)  which included a drawing tablet which could interpret a shape as control for sound synthesis. (metasynth) He used it to compose \textit{La L\'egende d'Eer} (1977). 

Subotnick again - with IRCAM and then with MIT and Barry Vercoe looked at the Performer in electronic music.  Vercoe with his score following (pitch detection) - if instrument does this, computer does that.  Subotnick developed Interactor.  Tod Machover - \textit{Valis} at IRCAM in 1986/1987 - it's in the library - worked on the principle of the hyperinstrument.  More and more detail was sampled from the instrument itself.

Of the most well known is Michel Waisvisz in Amsterdam. (page 228) worked with Johan den Biggelaar in 1984 to build the Hands  at STEIM.  They were connected up to the sensorlab a multiple control voltage to midi converter. Mattel Toys invented the PowerGlove.  
From that, and by watching the simultaneous hand movements of Indian singers, Laetitia Sonami in the early 90s found herself commissioning the Lady's Glove built by Bert Bongers. (p229) Radio Baton of Max Mathews and the Radio Drums used by Neil Rolnick.
David Rokeby in Toronto built the Very Nervous System which analyses video.  Motion tracking is now BIG BUSINESS. And the components have just got smaller and smaller. \textbf{Introduce the Arduino}

Kaija Saariaho - Finnish composer of note who spent (and spends) a lot of time at IRCAM (see Born).

Synthesis algorithms - physical models - pages 253 onwards. AM, FM, Granular - where one can work in realtime.  This is TRUAX here - page 257.  He made the PODX software and created such pieces as \textit{Wave edge} (1983), \textit{Solar Ellipse} (1985) and \textit{Riverrun} (1986) based on the idea ``on the flow of a river from the smallest droplets or grains to the magnificence, particularly in British Columbia of rivers…''


Software: Cmix - lansky in 1986. 1985 as a translation of the Music N programs to personal computers Barry Vercoe finished the first version of Csound.

1988 Jean-Francois Allouis became technical director at IRCAM needed to design a successor to the 4X. They chose the Next machine. The ISPW ran inside the Next. Used for Philippe Manoury's \textit{En Echo} (1984) and Pierre Boulez, \textit{`...explosante-fixe...'} (1994). Cort Lippe's music used a lot of the functionality of the 4x programming within the ISPW.  Shortly after the ISPW became available, Next stopped making computers.

Then the Giuseppe Di Guiigno connection - IRCAM - 4X - Bontempi - = MARS station. (Musical Audio Research Station) managed by Sylvian Sapir (who had graduated from a Phd in wave propogation in plastic: for anecdotes see P264).  MARS had a reputation but it really didn't extend very far.

In the US - Carla Scaletti (who had worked with Moog and buchla in the 70s)  and Kurt Hebel were working at CERL (Computer-based Education Research Laboratory) at the University of Illinois at Urbana in 1981.  They formed a partnership (company and a marriage) around a DSP called Platypus. Which in the end  turned into the Kyma system (running the Cabybara software).


