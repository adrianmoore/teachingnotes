%\chapter{USSS toolkits}
%shown in figure~\ref{fig:artificialnatural}
%pages~\pageref{tab:opposites1}
%and form (section~\ref{section:form}, page~\pageref{section:form}).

\chapter{Electronic music 7}
\label{history7}

\section{Electronica}
By the mid 1990s, the term electronica was a term used to refer to electroacoustic dance music. This is usually produced by a DJ/performer using some combination of turntables, synthesizers, samplers, and a variety of sound processing hardware such as effects processors. Groove boxes combine many of these features. Groove boxes facilitate new levels of beat construction, taking it to a sophisticated level of artistry by allowing short rhythmic patterns to be created on a grid, with each row corresponding to a different rhythmic instrument. Rhythms may be created in short order and looped to create repeating patterns. A variety of such patterns may be held in memory, and a DJ can have one play for an extended period of time as the basis of effects played on a synthesizer by the DJ or to underlie other effects such as turntable scratches. Extended, hypnotic pieces can be created by switching among rhythmic patterns and changing "melodic" elements to contrast the consistency of the looping rhythm. Alternatively, samples may be taken from classic albums and used as instruments. A measure of a classic drum part may be looped, or a programmed rhythmic pattern may be punctuated with a classic guitar riff. Some groove boxes also have oscillators that produce classic analog synthesis sounds. The oscillators may be pitched, so that simple melodies can be produced along with the rhythms. Some groove boxes also allow the tempo to be varied interactively via a "tap tempo" mode, by which the DJ taps a pad on the instrument. Some models also feature infrared light sources that affect parameters such as tempo, cutoff frequency, or filter resonance. Hand or body movements within the infrared beam can affect these parameters. DJs sometimes work in conjunction with other musicians, such as a percussionist or guitarist.
Electronica serves as a cultural history of popular musics that may be blended at will in the hands of a skilled DJ. It can be described as postmodern given its emphasis on collage representation and quotation.

Most of electronica's celebrities are known by nicknames only. Albums often feature no photographs or artwork. As such a variety of sources is present in this melting pot of syncopation and quotation, it is perhaps appropriate that no one individual is held responsible for the product in a conventional celebrity sense. This level of anonymity also derives from the environment of the "rave" event, in which the audience is the performance -- the performer is not meant to be the center of attention, but function instead to channel the energy of the crowd and provide an appropriate backdrop for the social event. Distinctions abound, however, in the plethora of identifiable styles that have emerged. Electronica exists in four broad genres, all of which grew out of various popular music styles dating back to the 1960s.

\textbf{House}This category forms the basis of many of the others. It consists of groups of 4 beats at about 60 beats per minute (about the typical rate of the human heart). There is typically a snare drum or hand clap on beats 2 and 4. House is also characterized by samples from excerpts from popular songs from the past.

This pattern grew out of the \textbf{Disco} movement of the 1970s, which, in addition to this rhythm pattern, also featured light and active bass lines that were meant to keep people moving on a dance floor. An outgrowth of Disco circa 1979 was HI NRG, which was more electronic and at a faster tempo. By 1980 there was a further outgrowth called Italo (originating in Italy), that was completely electronic.

In the mid 1980s, grassroots dance parties were popular in Chicago. The term House came from a venue called the Warehouse, where they were often held. Chicago House featured the Disco rhythmic pattern, and typically added a piano loop as well as vocals. A variant of this, \textbf{Deep House}, and a New York counterpart, \textbf{Garage}, used vocals with gospel influence, to create a more soulful sound.

At about the same time, another Chicago movement arose known as \textbf{Acid}. This featured psychedelic sounds created by a synthesizer (the Roland 303 became a staple of many DJs) that created the sound of classic analog waveforms and filter sweeps.

In about 1990 House became a commercial phenomenon in Europe. \textbf{Euro}, or \textbf{Club}, had a more mellow, polished and produced sound to it, with some of the edge of Chicago House taken out.

By 1997, Paris clubs had a form that might be called \textbf{French House}, which was essential Euro with a bit more grunge added. Speedgarage originated in London, and was a combination of House rhythms with a slow bass line.

\textbf{Techno} was also based on a four-beat pattern, Techno is somewhat faster and edgier than House. It does not feature the hand claps the way House does, and tends to focus more on synthesizer and recorded sound effects for a more mechanical and futuristic sound.
This style is an outgrowth of the machine sound originated by Kraftwerk in the 1970s. What is now called Techno came from Detroit. By 1985, a style called \textbf{Detroit Techno} emerged that featured fast, syncopated rhythms over the House/Disco four-beat pattern. A contrasting movement by 1989 was \textbf{Minimal Techno}, which featured a simple rhythm with as little else as possible on top of it.

By 1990, a more aggressive form had emerged called \textbf{Hardcore}. This was much faster, emphasizing frenetic rhythmic speeds bordering on complete disorder. An even faster style was \textbf{Gabber}, which featured comic, cartoonish samples.

By 1993, \textbf{Trance} music was Techno at high speeds, although not as fast as hardcore and Gabber. Hi-hat cymbal sounds are prominent, giving this a more sizzling feel. Ambient, spacy washes (an outgrowth of Brian Eno's earlier work) are also featured. Trance is meant to invoke a spiritual, trance-like state in the dancers, analogous to the spiritual trances experienced by shamans in some cultures. Even more psychedelic was Goa Trance, which featured vocals and even more synthesizer washes.

Also in the early 1990s, some producers started to create ``chill out music'' to be played as the party was breaking up. Intelligent Techno featured high degrees of electronic washes, but without the strong rhythmic component of dance music. It is often similar to Eno's ambient albums.

\textbf{Jungle} style is based in funk and soul music from the late 1960s and 1970s. is faster still, but with a less even drum and bass beat. Regular syncopations break up the regular patterns found in House and Techno. Jungle came about in the mid 1990s. An important series of sub-styles are known collectively as \textbf{Drum 'n' Bass}. It tends to combine lush, spacy washes with lighter bass lines reminiscent of bopping reggae basses, with predominant polyrhythmic percussion.

\textbf{Breakbeat} originated in England in the mid 1990s. It was based on syncopated sections of James Brown recordings, and represented an alternative to the ``four on the floor'' regularity of House and Techno.

Breakbeat grew out of \textbf{Hip Hop}, which originated in the 1970s and went through a ``revival'' in the late 1980s. Hip Hop was characterized by upbeat, rapped vocals and turntable scratching.

\textbf{Trip Hop} was a slowed down, listening version of Breakbeat, sometimes called ``dance music for a chair.''

In contrast, \textbf{Big Beat} is a heavier, faster sound, featuring screaming electronics

\section{The now}

Where are we now? Multiple speakers, tools, interactive, live coding. 
But what has changed?

Chadabe - p324 - where are we going.  CD-ROM works, World Wide Web??  Spectacles of sound and light after Xenakis \textit{Polytope de Cluny} (1972) and \textit{Diatope} (1978) - with \textit{La Légende d'Eer} (1977) as music for a spectacle that first presented at Centre Pompidou in Paris.  Sound artists and the idea of creating spaces - + interactivity in galleries or the environment??  Mathews' baton was made on the pretence of being able to buy the `score' and play through it `at your own pace'.  Morton Subotnick's \textit{All my hummingbirds have alibis} (1993) for multimedia CDROM allows the listener to select the sections.  Peter Gabriel's \textit{Xplora 1} alows one to manipulate the faders on a virtual mixer.  The whole net culture ? what do we do on the net ?

The new musical instruments, `as we learn how to make it, we'll learn how to play it' - or will (should ) it be the other way around.








