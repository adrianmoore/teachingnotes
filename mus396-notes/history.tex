%\chapter{USSS toolkits}
%shown in figure~\ref{fig:artificialnatural}
%pages~\pageref{tab:opposites1}
%and form (section~\ref{section:form}, page~\pageref{section:form}).

\chapter{History of Electronic Music}
\label{history}

\section{Pre-history 1}
Where to start? Music, past and present? It does appear that electroacoustics (and by this term let's say anything involving electronics or computers) was trying to expand the palette of colour available to the composer. Many novel inventions (often in the guise of instruments) produced clear pitches that allowed them to be played in a classical manner but many more, produced new sounds of completely unknown origins. So much so that it has changed the way we perceive electronic sounds structured to form an artwork. Note I studiously avoid the word music! We might reflect that in this day and age you might well have an online (sonic) artwork in one window running in Flash whilst you finish your email! The whole conception and consumption of art is in a state of flux.

There are so many milestones in the development of electroacoustic music throughout the last 50 years that we will not be able to cover them all. However, whilst technology has developed in a very streamlined, linear, ever-onward fashion (only going retrograde by emulation or fashion), musical developments that have taken place are not so easily charted. In no way can one tie the sonic artworks of the time to any technology and we should not try to do so.

Grove Dictionary
\begin{itemize}
\item \href{http://www.grovemusic.com/shared/views/article.html?from=search&session_search_id=1011869954&session_name=e09647b5a66fbfeb&hitnum=1&section=music.08694&start=1&query=electronic%20instruments&search_subview=search_subject}{electronic instruments}
\item \href{http://www.grovemusic.com/shared/views/article.html?section=music.08695}{electroacoustic music}
\item \href{http://www.grovemusic.com/shared/views/article.html?section=music.40583}{computers and music}
\end{itemize}

\begin{itemize}
\item Changes in the way we listened to music - Var\`ese.
\item Futurists - Russolo
\item Thaddeus Cahill - Telharmonium
\item The Theremin - OHM Rockmore 
\item Methods of composition - types that existed then and exist now (
Tape,
Mixed,
Live electronics,
Computer Aided,
Computer Music)
\end{itemize}

Changes in music appreciation.

\begin{itemize}
\item \href{http://www.grovemusic.com/shared/views/article.html?from=search&session_search_id=1013431856&session_name=f64e02b451f7a922&hitnum=1&section=music.00997&start=1&query=antheil&search_subview=search_subject}{Antheil} \textit{Ballet Mechanique} (1923-5)
\item \href{http://www.grovemusic.com/shared/views/article.html?from=search&session_search_id=1013432477&session_name=f64e02b451f7a922&hitnum=1&section=music.40105&start=1&query=satie&search_subview=search_subject}{Satie} \textit{Parade} (1917)
\item \href{http://www.grovemusic.com/shared/views/article.html?section=music.47335}{Respighi} \textit{Pines of Rome} (1924)
\item \href{http://www.grovemusic.com/shared/views/article.html?section=music.49908}{Cage} \textit{Imaginary Landscpes} (1939)
\item \href{http://www.grovemusic.com/shared/views/article.html?section=music.29042}{Var\`ese} \textit{Euqatorial} (1934), \textit{D\'eserts} (1950-54 ), \textit{Poem Electronique} (1958),
\end{itemize}


Futurists and Futurism:
\begin{itemize}
\item Filippo Tommaso Marinetti (1876-1944)
\item \href{http://www.grovemusic.com/shared/views/article.html?section=music.24174}{Luigi Russolo} (1885-1947)
\item \href{http://www.grovemusic.com/shared/views/article.html?from=search&session_search_id=1013432510&session_name=f64e02b451f7a922&hitnum=4&section=music.22259&start=1&query=marinetti&search_subview=search_subject}{Balilla Pratella} (1880-1965)
\end{itemize}

\begin{itemize}
\item Thaddeus Cahill (1867 - 1934) --- \href{http://www.grovemusic.com/shared/views/article.html?section=music.46183}{Telharmonium} 1897
\item \href{http://www.grovemusic.com/shared/views/article.html?section=music.45834}{Leon Theremin} (Lev Termen) (b St Petersburg, 15 Aug 1896; d Moscow, 3 Nov 1993)
\end{itemize}

\section{Pre-history 2}
\begin{itemize}
\item Electronic music developments in science and music
\item Schoenberg (1874-1951) - 5 orchestral pieces(1909) - III: Farben - Klangfarbenmelodie
\item Messiaen and the...Turanaglîla
\item Ondes Martenot - F\^te des Belles Eaux
\item Early Recording
\item Pierre Schaeffer in RTF - Cinq \'etudes de Bruits (1948) - \'Etudes aux objets (1959) - solfége
\item Jean Barraqu\'e - Etude (1953), Herbert Eimert + Robert Beyer - Klangstudie II (1952)
\item Stockhausen in RTF - Etude (1952)
\item Karlheinz Stockhausen in WDR - Electronishe Studies (1953, 1954), Gesang der J\"unglinge (1956), Kontakte (1960), Telemusik (1966), Hymnen (1966-7).
\end{itemize}

Schoenberg (writing to Mahler) ``..the possibility of creating a melody from one note played successively on different instruments''.

Messiaen (1908 - 1992): Oraison from 1937 is an extract from F\^ete des Belles Eaux. Commissioned and written for the world exposition in 1937

\begin{figure}[H]
\centering
\includegraphics[scale=0.6]{illustration}\caption{world fair 1937}
\label{fig:worldfair}
\end{figure}

A Sons et lumi\'ere performance with fireworks and water jets. Twenty composers were commissioned. Many wrote orchestral works but Messiaen wrote for a sextet of Ondes Martenots. The music was amplified by loudspeakers placed buildings on the banks of the river Seine.

What does the ondes martenot do / look like?

\begin{figure}[H]
\centering
\includegraphics[scale=0.6]{mart}\caption{ondes martenot 1}
\label{fig:worldfair}
\end{figure}

\begin{figure}[H]
\centering
\includegraphics[scale=0.6]{martenot}\caption{ondes martenot 2}
\label{fig:worldfair}
\end{figure}

\subsection{Early Recording}
\begin{itemize}
\item Thomas Edison 1877, phonograph (hollow cylinder)
\item Emile Berliner 1887, flat disc gramophone
\item In 1925 Kurt Stille and partners licensed magnetic tape production to Ludwig Blattner and Kaul Bauer
\item In 1930 Marconi purchased Blattner's.
\item In Germany Fritz Pfleumer interested I.G.Farben in developing plastic backed tape (much lighter and safer) and Allgemeine Electrizitats Gesellsch\"ft (AEG) in developing machines. [by 1945 AEG had 15khz frequency response in magnetic tape at 30ips). Then the Minnesota Mining and Manufacture company developed a new tape (3M).
\end{itemize}

Sound in film - optical recordings had been around for some time and composers including the British composer Daphne Oram used this technique.

\section{Schaeffer (RTF) and Stockhausen (WDR)}

Schaeffer (1910-1995) and the GRM.

Studio d'Essai under German occupation 1942 / Club d'Essai in 1946. Groupe de Recherches Musicales (GRM) in 1958.

\begin{itemize}
\item Cinq \'etudes de Bruits (1948) - Concert de Bruits 1948
\item Symphonie pour un homme seul (1950) - with Pierre Henry (1927).
\item Etudes aux objets (1959)
\end{itemize}

\subsection{The Cologne studios}

Herbert Eimert (1897 - 1972), Robert Beyer (1901-1989) - Klangstudie II

\begin{enumerate}
\item demonstrate the analysis and synthesis of timbres:...
\item establish finely graduated scales between tone and noise
\item demonstrate the unity of musical time by creating graduated transitions between pitch and rhythm
\item use differentiated scales of loudness and reverberation to create a multi-layered spatial perspective.
\end{enumerate}

\begin{itemize}
\item 1952 Konkrete Et\"ude (concrete music) - Stockhausen working with Schaeffer and studying with Messiaen
\item 1953 Elektronische Studie I
\item 1954 Electronische Studie II
\item 1955-6 Gesang der J\"nglinge
\item 1959-60 Kontakte (electronic and version for piano, percussion and electronics)
\item 1966 Telemusic (electronic music)
\item 1966 Hymnen (electronic and concrete music) - version with soloists - 1969 version with orchestra
\end{itemize}

