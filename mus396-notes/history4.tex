%\chapter{USSS toolkits}
%shown in figure~\ref{fig:artificialnatural}
%pages~\pageref{tab:opposites1}
%and form (section~\ref{section:form}, page~\pageref{section:form}).

\chapter{Electronic music 4}
\label{history4}

\section{Live Electronics, Stockhausen, Process music - Reich, Lucier, Tudor }
\begin{itemize}
\item Steve Reich - Come Out, It's Gonna Rain, Pendulum Music
\item Stockhausen MikrophonieI and II, Mantra
\end{itemize}

\textbf{Steve Reich}

\textit{It's Gonna Rain} 1965: \textit{Come Out} 1966,  which spawned Violin Phase and Piano Phase

\textit{Pendulum Music} 1968

Stockhausen \textit{Mikrophonie I} 1964-5

After a concrete etude by Henry in 1950 which used a tamtam excited by various objects. Mikrophonie I has two teams of players standing either side of the tamtam. Player 1 excites the tamtam. Player 2 uses the microphone to closely monitor what is going on. Player 3 operates the filter. The work is composed in 33 moments organised by a 'connection scheme'. The problem of notation is that it is more difficult too describe what to do than to describe the kind of sound required.

\textit{Mixtur} 1964-7 for five orchestral groups, 4 ring modulators and sine-wave generators. See \url{www.stockhausen.org/mixtur.html}

\textit{Mikrophonie II} 1965 for choir (6 sop, 6 bass), Hammond organ, 4 ring modulators and tape. The four mics and the outputs of the hammond organ are connected to the ring modulators thus fusing their signals. The text chosen was Einfach grammatische Meditationen by Helmut Heissenbüttel. Very similar opening to Gesang der Jünglinge - note quotes from earlier works on the tape.

Mantra 1970 for two pianists, short-wave radio receivers, Ring modulators + tone generators (modul B mudulation circuits). Chinese woodblocks, antique cymbals.

\begin{quotation}I was sitting next to the driver, and I just let my imagination completely loose. I was humming to myself…I hear this melody - it all came very quickly together: I had the idea of one single musical figure or formula that would be expanded over a very long period of time, and by that I meant fifty or sixty minutes. And these notes were the centres around which I'd continually present the same formula in a smaller form ….I wrote this melody down on an envelope''

(from Maconie and Kurtz)
\end{quotation}

To make electronic music live you put your processes on stage. Cage's \textit{Cartridge Music} (1960).
Cage set out a world of sound in which he freely explored.  In \textit{Child of Tree} (1975) Cage plucked spines from a cactus (amplified). 

Stockhausen: \textit{Mixtur} (1964), \textit{Mikrophonie II} (1965) for twelve singers, Hammond organ and ring modulator. \textit{Prozession} (1967), \textit{Stimmung} (1968), \textit{Aus den Sieben Tagen} (1968) and \textit{Mantra} (1970) for two pianos, percussion and ring modulators.

\subsection{USA}
San Francisco Tape Music Center : Pauline Oliveros and Ramon Sender plus, Terry Riley, and later Morton Subotnick.   Then Subotnick and Sender found a house on Jones Street and carried on. (see page 89 for good anecdotes of Oliveros, Subotnick, Chowning et al).
The tape music center moved to Mills College as the Rockerfeller Foundation needed an agent that was fiscally responsible for a large grant obtained by Subotnick. (1965-66)

Gordon Mumma – instrument building – working with Cunningham, and like Tudor, he built his own.

Alivin Lucier – \textit{Music for Solo Performer} (1965) used the sounds of a performer's alpha brain patterns.- these sounds then were used to resonate percussion instruments around the room.

\subsection{Electronic music and dance}.
Merce Cunningham used symphonie pour un homme seul in 1952  Then David Tudor (on commission from Cunningham) made Rainforest in 1958 where contact loudspeakers were attached to objects to make them resonate.    

see also the Maurice Bejart, Pierre Henry pieces.

The idea of putting a performer on stage – mixed or live music.  If one listens to well written tape and instrument, the synchronisation can be convincing despite the rigidity of the tape part.  Davidovsky \textit{Synchronisms 6} (1970) for piano and tape, enlarges the piano. Smalley's \textit{Piano Nets} does similar.  Cage's idea of simultaneity was to start two things off at the same time and see what happened. Thus his tape realisation of Fontana Mix (1958) is often played alongside a performance of Aria. In \textit{HPSCHD} (1969), seven harpsichordists play at their own tempos along with the playback of Lejaren Hiller's fifty-one tapes.  \textit{Roaratorio} (1979) is in that tradition: an ``irish circus''. Mesostics – Cage reading through his `Writing for the Second time through Finnegan's Wake', while several Irish musicians played irish folk music on traditional instruments and sixty-two channels of tape were played back conveying all of the sounds mentioned in Finnegans Wake. Conceived at IRCAM. 


