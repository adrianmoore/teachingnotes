%\chapter{USSS toolkits}
%shown in figure~\ref{fig:artificialnatural}
%pages~\pageref{tab:opposites1}
%and form (section~\ref{section:form}, page~\pageref{section:form}).

\chapter{Electronic music 4}
\label{history4}

\section{Live Electronics, Stockhausen, Process music - Reich, Lucier, Tudor }
\begin{itemize}
\item Steve Reich - Come Out, It's Gonna Rain, Pendulum Music
\item Stockhausen MikrophonieI and II, Mantra
\end{itemize}

\textbf{Steve Reich}

\textit{It's Gonna Rain} 1965: \textit{Come Out} 1966,  which spawned Violin Phase and Piano Phase

\textit{Pendulum Music} 1968

Stockhausen \textit{Mikrophonie I} 1964-5

After a concrete etude by Henry in 1950 which used a tamtam excited by various objects. Mikrophonie I has two teams of players standing either side of the tamtam. Player 1 excites the tamtam. Player 2 uses the microphone to closely monitor what is going on. Player 3 operates the filter. The work is composed in 33 moments organised by a 'connection scheme'. The problem of notation is that it is more difficult too describe what to do than to describe the kind of sound required.

\textit{Mixtur} 1964-7 for five orchestral groups, 4 ring modulators and sine-wave generators. See \url{www.stockhausen.org/mixtur.html}

\textit{Mikrophonie II} 1965 for choir (6 sop, 6 bass), Hammond organ, 4 ring modulators and tape. The four mics and the outputs of the hammond organ are connected to the ring modulators thus fusing their signals. The text chosen was Einfach grammatische Meditationen by Helmut Heissenbüttel. Very similar opening to Gesang der Jünglinge - note quotes from earlier works on the tape.

Mantra 1970 for two pianists, short-wave radio receivers, Ring modulators + tone generators (modul B mudulation circuits). Chinese woodblocks, antique cymbals.

\begin{quotation}I was sitting next to the driver, and I just let my imagination completely loose. I was humming to myself…I hear this melody - it all came very quickly together: I had the idea of one single musical figure or formula that would be expanded over a very long period of time, and by that I meant fifty or sixty minutes. And these notes were the centres around which I'd continually present the same formula in a smaller form ….I wrote this melody down on an envelope''

(from Maconie and Kurtz)
\end{quotation}

To make electronic music live you put your processes on stage. Cage's \textit{Cartridge Music} (1960).
Cage set out a world of sound in which he freely explored.  In \textit{Child of Tree} (1975) Cage plucked spines from a cactus (amplified). 

Stockhausen: \textit{Mixtur} (1964), \textit{Mikrophonie II} (1965) for twelve singers, Hammond organ and ring modulator. \textit{Prozession} (1967), \textit{Stimmung} (1968), \textit{Aus den Sieben Tagen} (1968) and \textit{Mantra} (1970) for two pianos, percussion and ring modulators.

\subsection{USA}
San Francisco Tape Music Center : Pauline Oliveros and Ramon Sender plus, Terry Riley, and later Morton Subotnick.   Then Subotnick and Sender found a house on Jones Street and carried on. (see page 89 for good anecdotes of Oliveros, Subotnick, Chowning et al).
The tape music center moved to Mills College as the Rockerfeller Foundation needed an agent that was fiscally responsible for a large grant obtained by Subotnick. (1965-66)

Gordon Mumma – instrument building – working with Cunningham, and like Tudor, he built his own.

Alivin Lucier – \textit{Music for Solo Performer} (1965) used the sounds of a performer's alpha brain patterns.- these sounds then were used to resonate percussion instruments around the room.

\subsection{Electronic music and dance}.
Merce Cunningham used symphonie pour un homme seul in 1952  Then David Tudor (on commission from Cunningham) made Rainforest in 1958 where contact loudspeakers were attached to objects to make them resonate.    

see also the Maurice Bejart, Pierre Henry pieces.

The idea of putting a performer on stage – mixed or live music.  If one listens to well written tape and instrument, the synchronisation can be convincing despite the rigidity of the tape part.  Davidovsky \textit{Synchronisms 6} (1970) for piano and tape, enlarges the piano. Smalley's \textit{Piano Nets} does similar.  Cage's idea of simultaneity was to start two things off at the same time and see what happened. Thus his tape realisation of \textit{Fontana Mix} (1958) is often played alongside a performance of Aria. In \textit{HPSCHD} (1969), seven harpsichordists play at their own tempos along with the playback of Lejaren Hiller's fifty-one tapes.  \textit{Roaratorio} (1979) is in that tradition: an ``irish circus''. Mesostics – Cage reading through his `Writing for the Second time through Finnegan's Wake', while several Irish musicians played irish folk music on traditional instruments and sixty-two channels of tape were played back conveying all of the sounds mentioned in Finnegans Wake. Conceived at IRCAM. 

\subsection{Conceptualism}
Conceptualism was a movement in the 1960s that pushed the boundaries of art. It was spurred in part by the re-evaluations of cultural assumptions that were rampant in this decade. The 1950s had been a decade of rebuilding all over the world after World War II. In America, this brought about some bizarre contrasts. America had emerged from the war as the dominant world power. Industry had been in high production throughout the war. As the war had not been fought in the US, this left America virtually unmatched in the ability to produce. Furthermore, many families had suddenly had two incomes during the war, as men who enlisted were receiving a salary, while at home wives also earned salaries working to sustain the war effort. The result was that after the war, more Americans had more disposable income than ever before. There was a feeling of confidence and prosperity, while at the same time there was an underlying absence of security. The Cold War had set in almost as soon as World War II had ended, and the threat of nuclear attack was prevalent. While people revelled in their new possessions and wealth, they also built bomb shelters in their back yards in case of attack. There was also a sense of suspicion underlying the sense of freedom and affluence. Senator Joseph McCarthy headed up the House of Un-American Activities that sought to root out communism from America. Many people were called to testify regarding suspected involvement in communist activities, and to give names of others who participated. President Eisenhower, who had commanded the US forces in Europe during the war, represented a continuing trend of uniformity and discipline.

This all started to unravel during the 1960s. John Kennedy was elected president at the age of 43, the youngest man ever to have been elected president. An age of youth and new possibilities was born. At the same time, the burgeoning civil rights movement, the Bay of Pigs, the Cuban Missile Crisis, the violent death of Kennedy, and the subsequent military confrontation in Vietnam, seemed to rupture the veneer of comfort and uniformity of the 1950s. Now, all bets were off. Everything was open to question. The nature of war, marriage, sex, clothing, education, all were up for grabs. Timothy Leary, a Harvard professor and proponent of LSD as a tool for awakening the consciousness, encouraged America's youth to "Tune in, turn on, and drop out." At the University of California Berkeley, the Physics Department was feeling the political heat as well. Scientists were made to give oaths of loyalty and to testify before the House of Un-American Activities. In 1964, the UC Chancellor banned politicking on the Berkeley campus, bringing about a series of protests and causing many scientists to abandon their research and become social activists.

Part and parcel of this time of upheaval and questioning was a re-examination of the nature of art and music. Much of this was an outgrowth of the currents originated by the experiments of John Cage, in which the performer's intention and ego were to be circumvented.
Many conceptualist pieces were that and only that: abstract concepts that were presented as ideas, which could in themselves be considered art. Word-pieces were meant to stimulate thought, and not necessarily be feasible as performances. La Monte Young's Composition 1960 no.15 consisted of: "This piece is little whirlpools out in the middle of the ocean." Young's Composition 1960 no.5 was: "Turn a butterfly (or any number of butterflies) loose in the performance area."

Cage had also stated that ``all music is theatre,'' and this idea was borne out in spades by a variety of performers who sought to blur the lines between music, theatre, and other arts. ``Happenings'' were interactive environments set up by artists, and the audience's interaction with the environment was the work of art. Many happenings and experimental music performances incorporated electronics as components of the experiment.

\subsection{Fluxus}

Started by graphic artist George Maciunas in about 1960, Fluxus was a deliberate revival of the anti-art credo of the dadaists. He was joined by composers La Monte Young, Richard Maxfield, and pianist and sculptor Yoko Ono. In time, many artists drifted through the Fluxus milieu. Fluxus pieces were meant to be actions or objects placed into new and ironic contexts. In George Brecht's Motor Vehicle Sundown (Event), people were to sit in their cars and follow directions written on notecards to honk their horns, turn their headlights on and off, or open and close their doors. Annea Lockwood's Piano Burning involved putting small contact microphones inside of a piano, splashing kerosene on it, and setting it on fire, amplifying the sound of the instrument burning.

\subsection{Merce Cunningham}

Choreographer Merce Cunningham (1919-2009), often as part of his long-term collaboration with John Cage, had begun to incorporate electroacoustic music into his performances as early as 1952, when he used Schaeffer and Henry's Symphonie pour un Homme Seul as music for a dance.

In 1953, David Tudor became one of the core musicians of Cunningham's company. Tudor, a pianist, specialized in creating his own electronic instruments that interacted with the dancers. In Rainforest (1958), for example, contact loudspeakers were attached to pieces of sculpture. The speakers produced electronic sounds at the resonant frequencies of the sculptures. Later versions included Cage's music played through the speakers.

In Cunningham's Variations V (1965), dancers triggered sound in as many ways as possible. Modified theremins, designed by Bob Moog, were placed around the stage on poles. Different sounds were triggered as dancers moved within the proximity of different poles. The musicians/composers included John Cage, James Tenney, David Tudor, and Gordon Mumma. They controlled the sounds and their durations as they were emitted from various objects around the stage -- a plant, a pillow, a table, two chairs. The objects had contact microphones attached, so when dancers touched or moved them, they also produced sounds from tape, oscillators, and shortwave radios. The piece ended with Cunningham riding about the stage on a bicycle that had sound produced from its wheels.
 
Gordon Mumma. In Ann Arbor, Michigan, a group of composers and artists founded the Once Group, which produced periodic Once Festivals from 1961 to 1968 of taped music, live electronics, theatre, dance, film, and opera. The festival's principal composers were Robert Ashley and Gordon Mumma. The Once Festivals became major gatherings for the weird and offbeat in all of the arts, attracting all kinds of experimentation and integration of unrelated arts and concepts.


\subsection{San Francisco Tape Music Center}
In San Francisco, a group of multi-media artists that included Pauline Oliveros, Ramon Sender, Morton Subotnick, Terry Riley, and Phil Winsor began to hold multi-media events that incorporated as many elements as possible -- amplified music, projections, and film. Ramon Sender's Tropical Fish Opera (1962) consisted of a fish in a tank swimming in front of a score, and the performers played the score by translating the fish's movements into notes or dynamics. City Scale (1963) involved seating audience members in the back of pickup trucks and driving them around town to different scenes that were being staged.

One of their big possessions was a three-track tape recorder that allowed them to do experiments with tape editing. They were soon joined by Don Buchla, a physicist and electrician. Buchla worked on a number of NASA projects at Berkeley before leaving the lab for the protests. He also worked on a number of projects involving sound. One was a transistorized hearing aid. Another project, an aid for blind people, was a device that emitted a pitch that changed according the proximity of objects. (Buchla lost interest in this project when the FBI started to become interested in it.) Buchla also worked with tape and created musique concrete pieces. After seeing a performance at the Tape Music Center and their three-track recorder, he became a fixture at the Center as well.

Among the pieces that premiered there were In C (1964) by Terry Riley, It's Gonna Rain (1965) by Steve Reich, and Desert Ambulance (1964) by Ramon Sender. The group also hosted performances by John Cage, David Tudor, and lectures by Karlheinz Stockhausen.

Alvin Lucier (1931-) attempted to incorporate biofeedback into performance and music. Most notable was Music for a Solo Performer, which featured amplified brain waves. A performer (Lucier) sat onstage as a technician connected electrodes to his head for several minutes. The performer sat quietly, and the brainwave signals were audified from an electroencephelogram (EEG). The EEG signals were amplified, and near the speakers were a variety of percussion instruments. The low frequency EEG instruments caused the instruments to vibrate. The performance ended when the performer opened his eyes.

