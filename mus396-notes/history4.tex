%\chapter{USSS toolkits}
%shown in figure~\ref{fig:artificialnatural}
%pages~\pageref{tab:opposites1}
%and form (section~\ref{section:form}, page~\pageref{section:form}).

\chapter{Electronic music 4}
\label{history4}

\section{Synthesizers and Computers}
\begin{itemize}
\item Steve Reich - Come Out / It's Gonna Rain / Pendulum Music
\item Stockhausen MikrophonieI and II, Mantra
\end{itemize}

\textbf{Steve Reich}

\textit{It's Gonna Rain} 1965: \textit{Come Out} 1966,  which spawned Violin Phase and Piano Phase

\textit{Pendulum Music} 1968

Stockhausen \textit{Mikrophonie I} 1964-5

After a concrete etude by Henry in 1950 which used a tamtam excited by various objects. Mikrophonie I has two teams of players standing either side of the tamtam. Player 1 excites the tamtam. Player 2 uses the microphone to closely monitor what is going on. Player 3 operates the filter. The work is composed in 33 moments organised by a 'connection scheme'. The problem of notation is that it is more difficult too describe what to do than to describe the kind of sound required.

\textit{Mixtur} 1964-7 for five orchestral groups, 4 ring modulators and sine-wave generators. See \url{www.stockhausen.org/mixtur.html}

\textit{Mikrophonie II} 1965 for choir (6 sop, 6 bass), Hammond organ, 4 ring modulators and tape. The four mics and the outputs of the hammond organ are connected to the ring modulators thus fusing their signals. The text chosen was Einfach grammatische Meditationen by Helmut Heissenbüttel. Very similar opening to Gesang der Jünglinge - note quotes from earlier works on the tape.

Mantra 1970 for two pianists, short-wave radio receivers, Ring modulators + tone generators (modul B mudulation circuits). Chinese woodblocks, antique cymbals.

\begin{quotation}I was sitting next to the driver, and I just let my imagination completely loose. I was humming to myself…I hear this melody - it all came very quickly together: I had the idea of one single musical figure or formula that would be expanded over a very long period of time, and by that I meant fifty or sixty minutes. And these notes were the centres around which I'd continually present the same formula in a smaller form ….I wrote this melody down on an envelope''

(from Maconie and Kurtz)
\end{quotation}
