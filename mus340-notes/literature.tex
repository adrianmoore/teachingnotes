%\chapter{USSS toolkits}
%shown in figure~\ref{fig:artificialnatural}
%pages~\pageref{tab:opposites1}
%and form (section~\ref{section:form}, page~\pageref{section:form}).

\chapter{Literature}
\label{literatures}

\begin{itemize}
\item David Lewis Yewdall's book \textit{Practical Art of Motion Picture Sound} \citep{yewdall2003practical} 
is the text if you want to go into sound recording, editing and mixing for film. It contains real examples of what worked and what didn't alongside very easy to read technical descriptions. 


\item Andy Farnell's book \textit{Designing Sound} \citep{farnell2010designing}

 
\item Nicholas Cook's book \textit{analysing musical multimedia} \citep{cook1998analysing} looks at commercials, CD-ROMs, music videos and songs, operas, ballet. Key to this book is the analysis of music as a component part of the multimedia setting. 


\item Michel Chion's book \textit{Audio-Vision: sound on screen} \citep{chion1990} 


\item Roy M. Prendergast's book \textit{Film Music: a neglected art} \citep{prendergast1992film} 

\item Karen Collins' book \textit{Game sound: an introduction to the history, theory, and practice of video game music and sound design} \citep{collins2008game}


\item Roy Thompson and Christopher J. Bowen's book \textit{Grammar of the Edit} \citep{thompson2009grammar} 

\item Richard Davis' book \textit{Complete Guide to Film Scoring} \citep{davis2010complete}

\item Mervyn Cooke's book \textit{A history of film music} \citep{cooke2008history}

And for orchestration and harmony you will need to reference

\item Samuel Adler's book on orchestration which clearly details ranges and comes with plenty of musical examples \citep{adler1989study} 
\item Walter Piston's excellent book on harmony \citep{piston1978harmony}
\item Taking the Piston further is Schoenberg's complete resource \citep{schoenberg1978theory}
\end{itemize}
