%\chapter{USSS toolkits}
%shown in figure~\ref{fig:artificialnatural}
%pages~\pageref{tab:opposites1}
%and form (section~\ref{section:form}, page~\pageref{section:form}).

\chapter{History of film music and sound}
\label{history}

\section{Historical trends and paths}
Looking at Cooke \citeyearpar{cooke2008history} and Prendergast \citeyearpar{prendergast1992film}.

We  begin with the silence of silent films.  
Sound used to cover up the noise of the projectors and further direct the audience's concentration.

Then the temptation to add your own `imaginary' sounds when watching very realistic events. 

Sergei Eisenstein \textit{The Battleship Potemkin} (1925). 

And this very act of `perceiving' sounds by watching them happen was used by early film makers as an effect. 

Remember that even in foley, room noise is always dubbed on to give `realism'.

Sometimes `silence' is created by focusing on a single motive rather than neglecting everything. Hitchcock's \textit{The Birds} (1963) 

/textit{we need to think about what music does and therefore ask, what is film without music?}

Another reason why music appeared alongside film was because of the absence of real sound. You had actors lip syncing live from behind a screen (very acousmatic). Music ultimately `made the film feel real'. 

(page 7) Emile Reynaud's animated \textit{Pantomimes lumineuses} with music by Gaston Paulin in November 1892 as the first film music. And Georges M\'eli\`es played the piano for his own \textit{Le Voyage dans la lune}in 1902. (see Hugo for a realisation). 

Cinemas grew in the early 20th Century. 
Diegetic - often light-hearted in nature. Non-diegetic much more melodramatic. Indeed, although we don't hear it, live music was often played on set to set the mood for the actors.  

Successful production companies (still extant today) include that set up by Charles Path\'e in the early 1900s. And pianists were given cue-sheets with specific pieces of music to accompany films. Rossini \textit{William Tell} for `hurry' scenes. 

A general paradox of film music (even to this day):
\begin{quotation}
If you come out of the theatre almost unaware of the musical accompaniment to the picture you have just witnessed, the work of the musical director has been successful. Without music the present day audience would feel utterly lost. With it they should obtain an added satisfaction from the show, and still remain unconscious of the very thing which has produced that satisfaction. 

\raggedleft{\citep[16]{cooke2008history}}
\end{quotation}

Ultimately some of the key psychological features of film music boil down to simple instructions which have applied for many years. `Soft for going to happen, loud for happening'. 

\subsection{Venues and ensembles}
The rise of dedicated venues for cinema gave rise to dedicated instruments (in particular the Wurlitzer organ) and ensembles. 

\subsection{Early Epics}
Including D.W. Griffith's civil war movie \textit{The Birth of a Nation} (1915) (YT)
At three hours, it's a real epic. Original music composed by Joseph Carl Breil.

\subsection{Music for comedy}
Charlie Chaplin, Harold Lloyd, Buster Keaton.
Charlie Chaplin - \textit{Tillie's Punctured Romance} (1914) around 24:30 (YT) inside a nickelodeon (from Nickel - the 5 cent coin and odeion, a roofed theatre space).

\subsection{Music specifically for film}
Erik Satie's \textit{Entr'acte} from 1924.

Music for Sergei Eisenstein's \textit{Battleship Potemkin} (1925). With original music by Edmund Meisel but normally heard with a selection of Shostakovich. 
But Shostakovich was key to the development of film music in Russia. He worked at movie theatres in Leningrad and composed set pieces for a number of important films of the 20s. 

\subsection{The soundtrack}
Whilst there was some resistance to the addition of real-time sound, it was inevitable. 

Al Jolson in Warner Brother's \textit{The Jazz Singer} (1927) was the first ``talkie'' using the vitaphone system (a vinyl disc linked to the projector).
But \textit{The Jazz Singer} still used a compilation score (mainly Tchaikovsky - brief examples on YT) 

%Take the Jaws theme and elaborate 
%Source music and score for a scene

\section{Hollywood}

Hollywood and the studio system made film making a commercial art. MGM, Paramount, Warner Bros, Twentieth-Century Fox and RKO (Radio-Keith-Orpheum). And working conditions for composers were stressful and demanding. 

Cueing and synch was completed using a click track. Then `ducked' to enable the speech to come through. This is vital to remember if ever scoring with dialogue. 

%practical example here
% look at opera briefly

The symphonic style was unashamedly romantic - possibly a throwback from opera's success at merging music and acting, and in America, from Broadway theatre. And from Wagner, the leitmotif. 

Examples. Star Wars, Indiana Jones, Jaws. 

Citing Gorbman: Hollywood's compositional principles.
\begin{quotation}

\begin{enumerate}
\item  \textit{Invisibility}: the technical apparatus of nondiegetic music must not be visible.
\item \textit{Inaudibility}: Music is not meant to be heard consciously. As such it should subordinate itself to dialogue, to visuals - i.e., to the primary vehicles of the narrative.
\item \textit{Signifier of emotion}: Soundtrack music may set specific moods and emphasize particular emotions suggested in the narrative, but first and foremost, it is a signifier of emotion itself.
\item \textit{Narrative cueing}: 
\begin{itemize}
\item \textit{referential/narrative}: music gives referential and narrative cues, e.g., indicating point of view, supplying formal demarcations, and establishing setting and characters.
\item \textit{connotative}: music `interprets' and `illustrates' narrative events.
\end{itemize}
\item \textit{Continuity}: music provides formal and rhythmic continuity - between shots, in transitions between scenes, by filling `gaps'.
\item \textit{Unity}: via repetition and variation of musical material and instrumentation, music aids in the construction of formal and narrative unity.
\item A given film score may violate an of the principles above, providing the vioilation is at the service of the other principles.

\end{enumerate}
\raggedleft{\citep[73]{gorbman1987unheard}}
\end{quotation}


\section{Key figures}
\begin{itemize}
\item Max Steiner - best known for huge score of \textit{King Kong} (1933) (YT and Naxos). Also for the inside/outside use of non/diegetic music in \textit{Casablanca} (1942) with the use of Herman Hupfeld's song `As Time Goes By' (from 1931) 
\item Erich Korngold - well known for Errol Flynn blockbusters such as \textit{The Adventures of Robin Hood} (1938)
\item Franz Waxman - well known for horror music in particular \textit{The Bride of Frankenstein} (1935). Also work with Hitchcock (such as \textit{Rebecca} (1940), \textit{Suspicion} (1941) and \textit{Rear Window} (1954))
\item Alfred Newman - \textit{The Prisoner of Zenda} (1940)
\item Aaron Copland - cf. \textit{Billy the Kid} (1938) for the cowboy feel. Heard briefly in \textit{Of Mice and Men} (1940) (YT)
\item Mikl\'os R\'osa - \textit{Ben Hur} (1959)
\item Elmer Bernstein - \textit{To Kill a Mockingbird} (1962)
\item David Shire - \textit{The Taking of Pelham 123} (1974)
\item Jerry Goldsmith - \textit{The Planet of the Apes} (1968) and \textit{Escape from the Planet of the Apes} (1971)
\item Louis and Bebe Barron - \textit{The Forbidden Planet} (1956)
\item Alex North - \textit{2001: A Space Odyssey} (1968) YT but here it is interesting to note how Stanley Kubrick became so enamoured with his temp tracks that North's score was purged. 
\item Bernard Herman - beginning with \textit{Citizen Kane} (1941) and following, a close working relationship with Hitchcock. \textit{Vertigo}(1958), \textit{North by Northwest}(1959) and \textit{Psycho}(1960). N.B. whistling tune from \textit{Twisted Nerve} (1968) used in....
and in the United Kingdom, none other than.

\item Ralph Vaughan Williams - \textit{Scott of the Antarctic}(1948)
\item Benjamin Britten - \textit{Night Mail} (1936) - documentary and information films, but here with film composed to music and text. (YT)
\item William Walton
\item Malcolm Arnold
\item Richard Rodney Bennett - \textit{Four Weddings and a Funeral}(1994)
\item John Barry - \textit{Dances with Wolves} (1990)
and in cartoons

\item Carl Starling - WB cartoons for Daffy Duck, Donald Duck, Bugs Bunny etc. (Duck Amuck, 1953,  Vimeo) 
\end{itemize}

The list is endless and world-wide. 


\section{Historical film music practice and analysis}

Looking at Chion \citeyearpar{chion1990} and Cook \citeyearpar{cook1998analysing}

Although sound and music should magically `meld' with the image, Chion talks about `added value'. 

