%\chapter{USSS toolkits}
%shown in figure~\ref{fig:artificialnatural}
%pages~\pageref{tab:opposites1}
%and form (section~\ref{section:form}, page~\pageref{section:form}).

\chapter{History of film music and sound}
\label{history}

\section{Historical trends and paths}
Looking at Cooke \citeyearpar{cooke2008history} and Prendergast \citeyearpar{prendergast1992film}.

We  begin with the silence of silent films.  
Sound used to cover up the noise of the projectors and further direct the audience's concentration.

Then the temptation to add your own `imaginary' sounds when watching very realistic events. 

Sergei Eisenstein \textit{The Battleship Potemkin} (1925). 

And this very act of `perceiving' sounds by watching them happen was used by early film makers as an effect. 

Remember that even in foley, room noise is always dubbed on to give `realism'.

Sometimes `silence' is created by focusing on a single motive rather than neglecting everything. Hitchcock's \textit{The Birds} (1963) 

/textit{we need to think about what music does and therefore ask, what is film without music?}

Another reason why music appeared alongside film was because of the absence of real sound. You had actors lip syncing live from behind a screen (very acousmatic). Music ultimately `made the film feel real'. 

(page 7) Emile Reynaud's animated \textit{Pantomimes lumineuses} with music by Gaston Paulin in November 1892 as the first film music. And Georges M\'eli\`es played the piano for his own \textit{Le Voyage dans la lune}in 1902. (see Hugo for a realisation). 

Cinemas grew in the early 20th Century. 
Diegetic - often light-hearted in nature. Non-diegetic much more melodramatic. Indeed, although we don't hear it, live music was often played on set to set the mood for the actors.  



\section{Historical film music practice and analysis}

Looking at Chion \citeyearpar{chion1990} and Cook \citeyearpar{cook1998analysing}

