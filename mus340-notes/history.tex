%\chapter{USSS toolkits}
%shown in figure~\ref{fig:artificialnatural}
%pages~\pageref{tab:opposites1}
%and form (section~\ref{section:form}, page~\pageref{section:form}).

\chapter{History of film music and sound}
\label{history}

\section{Historical trends and paths}
Looking at Cooke \citeyearpar{cooke2008history} and Prendergast \citeyearpar{prendergast1992film}.

We  begin with the silence of silent films.  
Sound used to cover up the noise of the projectors and further direct the audience's concentration.

Then the temptation to add your own `imaginary' sounds when watching very realistic events. 

Sergei Eisenstein \textit{The Battleship Potemkin} (1925). 

And this very act of `perceiving' sounds by watching them happen was used by early film makers as an effect. 

Remember that even in foley, room noise is always dubbed on to give `realism'.

Sometimes `silence' is created by focusing on a single motive rather than neglecting everything. Hitchcock's \textit{The Birds} (1963) 

\section{Historical film music practice and analysis}

Looking at Chion \citeyearpar{chion1990} and Cook \citeyearpar{cook1998analysing}

