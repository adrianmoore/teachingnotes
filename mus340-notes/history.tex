%\chapter{USSS toolkits}
%shown in figure~\ref{fig:artificialnatural}
%pages~\pageref{tab:opposites1}
%and form (section~\ref{section:form}, page~\pageref{section:form}).

\chapter{History of film music and sound}
\label{history}

\section{Historical trends and paths}
Looking at Cooke \citeyearpar{cooke2008history} and Prendergast \citeyearpar{prendergast1992film}.

We  begin with the silence of silent films.  
Sound used to cover up the noise of the projectors and further direct the audience's concentration.

Then the temptation to add your own `imaginary' sounds when watching very realistic events. 

Sergei Eisenstein \textit{The Battleship Potemkin} (1925). 

And this very act of `perceiving' sounds by watching them happen was used by early film makers as an effect. 

Remember that even in foley, room noise is always dubbed on to give `realism'.

Sometimes `silence' is created by focusing on a single motive rather than neglecting everything. Hitchcock's \textit{The Birds} (1963) 

/textit{we need to think about what music does and therefore ask, what is film without music?}

Another reason why music appeared alongside film was because of the absence of real sound. You had actors lip syncing live from behind a screen (very acousmatic). Music ultimately `made the film feel real'. 

(page 7) Emile Reynaud's animated \textit{Pantomimes lumineuses} with music by Gaston Paulin in November 1892 as the first film music. And Georges M\'eli\`es played the piano for his own \textit{Le Voyage dans la lune}in 1902. (see Hugo for a realisation). 

Cinemas grew in the early 20th Century. 
Diegetic - often light-hearted in nature. Non-diegetic much more melodramatic. Indeed, although we don't hear it, live music was often played on set to set the mood for the actors.  

Successful production companies (still extant today) include that set up by Charles Path\'e in the early 1900s. And pianists were given cue-sheets with specific pieces of music to accompany films. Rossini \textit{William Tell} for `hurry' scenes. 

A general paradox of film music (even to this day):
\begin{quotation}
If you come out of the theatre almost unaware of the musical accompaniment to the picture you have just witnessed, the work of the musical director has been successful. Without music the present day audience would feel utterly lost. With it they should obtain an added satisfaction from the show, and still remain unconscious of the very thing which has produced that satisfaction. 

\raggedleft{\citep[16]{cooke2008history}}
\end{quotation}

Ultimately some of the key psychological features of film music boil down to simple instructions which have applied for many years. `Soft for going to happen, loud for happening'. 

\subsection{Venues and ensembles}
The rise of dedicated venues for cinema gave rise to dedicated instruments (in particular the Wurlitzer organ) and ensembles. 

\subsection{Early Epics}
Including D.W. Griffith's civil war movie \textit{The Birth of a Nation} (1915) (YT)
At three hours, it's a real epic. Original music composed by Joseph Carl Breil.

\subsection{Music for comedy}
Charlie Chaplin, Harold Lloyd, Buster Keaton.
Charlie Chaplin - \textit{Tillie's Punctured Romance} (1914) around 24:30 (YT) inside a nickelodeon (from Nickel - the 5 cent coin and odeion, a roofed theatre space).

\subsection{Music specifically for film}
Erik Satie's \textit{Entr'acte} from 1924.

Music for Sergei Eisenstein's \textit{Battleship Potemkin} (1925). With original music by Edmund Meisel but normally heard with a selection of Shostakovich. 
But Shostakovich was key to the development of film music in Russia. He worked at movie theatres in Leningrad and composed set pieces for a number of important films of the 20s. 

\subsection{The soundtrack}
Whilst there was some resistance to the addition of real-time sound, it was inevitable. 

Al Jolson in Warner Brother's \textit{The Jazz Singer} (1927) was the first ``talkie'' using the vitaphone system (a vinyl disc linked to the projector).
But \textit{The Jazz Singer} still used a compilation score (mainly Tchaikovsky - brief examples on YT) 

%Take the Jaws theme and elaborate 
%Source music and score for a scene

\section{Hollywood}

Hollywood and the studio system made film making a commercial art. MGM, Paramount, Warner Bros, Twentieth-Century Fox and RKO (Radio-Keith-Orpheum). And working conditions for composers were stressful and demanding. 

Cueing and synch was completed using a click track. Then `ducked' to enable the speech to come through. This is vital to remember if ever scoring with dialogue. 

%practical example here
% look at opera briefly

The symphonic style was unashamedly romantic - possibly a throwback from opera's success at merging music and acting, and in America, from Broadway theatre. And from Wagner, the leitmotif. 

Examples. Star Wars, Indiana Jones, Jaws. 

Citing Gorbman: Hollywood's compositional principles.
\begin{quotation}

\begin{enumerate}
\item  \textit{Invisibility}: the technical apparatus of nondiegetic music must not be visible.
\item \textit{Inaudibility}: Music is not meant to be heard consciously. As such it should subordinate itself to dialogue, to visuals - i.e., to the primary vehicles of the narrative.
\item \textit{Signifier of emotion}: Soundtrack music may set specific moods and emphasize particular emotions suggested in the narrative, but first and foremost, it is a signifier of emotion itself.
\item \textit{Narrative cueing}: 
\begin{itemize}
\item \textit{referential/narrative}: music gives referential and narrative cues, e.g., indicating point of view, supplying formal demarcations, and establishing setting and characters.
\item \textit{connotative}: music `interprets' and `illustrates' narrative events.
\end{itemize}
\item \textit{Continuity}: music provides formal and rhythmic continuity - between shots, in transitions between scenes, by filling `gaps'.
\item \textit{Unity}: via repetition and variation of musical material and instrumentation, music aids in the construction of formal and narrative unity.
\item A given film score may violate an of the principles above, providing the vioilation is at the service of the other principles.

\end{enumerate}
\raggedleft{\citep[73]{gorbman1987unheard}}
\end{quotation}


\section{Key figures}
\begin{itemize}
\item Gottfried Huppertz - best known for his score to Fritz Lang's \textit{Metropolis} (1927). The story line has its Wagnerian aspects (Sci-fi meets Parsifal) and the score is very reminiscent of Wagner, Strauss and Mahler. Note the leitmotives and use of pre-existing themes such as the \textit{Dies irae} and \textit{Marseillaise}. Very descriptive writing and it's always the music which pre-figures the action and sets the scene. 
\item Max Steiner - best known for huge score of \textit{King Kong} (1933) (YT and Naxos). Also for the inside/outside use of non/diegetic music in \textit{Casablanca} (1942) with the use of Herman Hupfeld's song `As Time Goes By' (from 1931) 
\item Erich Korngold - well known for Errol Flynn blockbusters such as \textit{The Adventures of Robin Hood} (1938)
\item Franz Waxman - well known for horror music in particular \textit{The Bride of Frankenstein} (1935). Also work with Hitchcock (such as \textit{Rebecca} (1940), \textit{Suspicion} (1941) and \textit{Rear Window} (1954))
\item Alfred Newman - \textit{The Prisoner of Zenda} (1940)
\item Aaron Copland - cf. \textit{Billy the Kid} (1938) for the cowboy feel. Heard briefly in \textit{Of Mice and Men} (1940) (YT)
\item Mikl\'os R\'osa - \textit{Ben Hur} (1959)
\item Elmer Bernstein - \textit{To Kill a Mockingbird} (1962)
\item David Shire - \textit{The Taking of Pelham 123} (1974)
\item Jerry Goldsmith - \textit{The Planet of the Apes} (1968) and \textit{Escape from the Planet of the Apes} (1971)
\item Louis and Bebe Barron - \textit{The Forbidden Planet} (1956)
\item Alex North - \textit{2001: A Space Odyssey} (1968) YT but here it is interesting to note how Stanley Kubrick became so enamoured with his temp tracks that North's score was purged. 
\item Bernard Herman - beginning with \textit{Citizen Kane} (1941) and following, a close working relationship with Hitchcock. \textit{Vertigo}(1958), \textit{North by Northwest}(1959) and \textit{Psycho}(1960). N.B. whistling tune from \textit{Twisted Nerve} (1968) used in....Herman in Cook \citep[66]{cook1998analysing},
\begin{quotation}
I feel that music on the screen can seek out and intensify the inner thoughts of the characters. It can invest a scene with terror, grandeur, gaiety, or misery. It can propel narrative swiftly forward, or slow it down. It often lifts mere dialogue into the realm of poetry. Finally, it is the communicating link between the screen and the audience, reaching out and enveloping all into one single experience.
\end{quotation}

and in the United Kingdom, none other than.

\item Ralph Vaughan Williams - \textit{Scott of the Antarctic}(1948)
\item Benjamin Britten - \textit{Night Mail} (1936) - documentary and information films, but here with film composed to music and text. (YT)
\item William Walton
\item Malcolm Arnold
\item Richard Rodney Bennett - \textit{Four Weddings and a Funeral}(1994)
\item John Barry - \textit{Dances with Wolves} (1990)
and in cartoons

\item Carl Starling - WB cartoons for Daffy Duck, Donald Duck, Bugs Bunny etc. (Duck Amuck, 1953,  Vimeo) 
\end{itemize}

The list is endless and world-wide. 

\section{Sound of Cinema (BBC)}
\subsection{episode 1}
After the silents and films such as \textit{Don Juan} (1926) with discs synchronised with the picture you get works like \textit{King Kong} (1933) fully scored by Max Steiner adopting the leitmotif. For composers like Erich Wolfgang Korngold \textit{Captain Blood, 1934} the studios gave him full control over the film process. \textit{The Adventures of Robin Hood} (1938) is another prime example of his work. Korngold worked for Warner Bros. 
The BBC programme lingered on the work of Bernard Hermann naturally with \textit{Citizen Kane} then the Hitchcock collaborations: \textit{Vertigo}(1958), \textit{Psycho}(1960), \textit{Marnie}(1964) - a failure, \textit{Torn Curtain}(1965) - the film that got Hermann fired. In 1975 Hermann worked with Martin Scorsese on \textit{Taxi Driver}.

\section{Well known films and their composers}
 
\begin{table}[H]
\begin{tabular}{|p{5.0cm}|p{8.0cm}|}
\hline
Film & Composer \\\hline
John Barry &  From Russia with Love   \\\hline
John Barry &  The Ipcress File  \\\hline
John Barry &  The Quiller Memorandum   \\\hline
John Barry &  Octopussy   \\\hline
John Barry &  Midnight Cowboy   \\\hline
John Barry &  Dances with Wolves   \\\hline
John Barry &  Out of Africa   \\\hline
John Barry &  The Specialist   \\\hline
John Barry &  Mercury Rising   \\\hline
Jerry Goldsmith &  The Man from UNCLE   \\\hline
Jerry Goldsmith &  The Waltons   \\\hline
Jerry Goldsmith &  Papillon   \\\hline
Jerry Goldsmith &  The Omen   \\\hline
Jerry Goldsmith &  Alien   \\\hline
Jerry Goldsmith &  First Blood   \\\hline
Jerry Goldsmith &  Total Recall   \\\hline
Jerry Goldsmith &  Forever Young   \\\hline
Jerry Goldsmith &  Star Trek: Nemesis   \\\hline
John Williams & Close Encounters of the Third Kind    \\\hline
John Williams & Raiders of the Lost Ark    \\\hline
John Williams & ET: The Extra Terrestrial    \\\hline
John Williams & Schindler's List    \\\hline
John Williams & Saving Private Ryan    \\\hline
John Williams & Minority Report    \\\hline
John Williams & Catch me if you can    \\\hline
John Williams & Harry Potter    \\\hline
John Williams & Star Wars    \\\hline
Danny Elfman & Spider Man    \\\hline
Hans Zimmer &  Gladiator   \\\hline
Howard Shore &  The Lord of the Rings   \\\hline
Vangelis &  Chariots of Fire   \\\hline
James Horner & Titanic    \\\hline
Alan Silvestri & Forest Gump    \\\hline

\end{tabular}
\label{tab:composersandfilms}
\end{table}

\section{Historical film music practice and analysis}

Looking at Chion \citeyearpar{chion1990} and Cook \citeyearpar{cook1998analysing}

Although sound and music should magically `meld' with the image, Chion talks about `added value'. 

Where the music is very closely linked in style, emotion and meaning it is `empathetic'. Where this is not the case the music can be `anempathetic'. 
\citep[14-15]{chion1990} gives us examples of how sound `temporalizes' image.
\begin{itemize}
\item Sustained sounds versus fluttering sounds (one is clearly more animated)
\item Predictability versus irregularity. 
\end{itemize}

Sound can augment and suggest what we're seeing (or as in horror movies) what we're not seeing. 

\section{Three modes of listening}
\begin{itemize}
\item Causal listening: `what is it?'
\item Semantic listening: `assuming a language, what is it saying?'
\item Reduced listening: `what is in the sound itself?'
\end{itemize}

Question: What do you have to work with?
\begin{itemize}
\item The sound recorded with the film
\item Foley
\item Dialogue / Dialogue replacement
\item Sfx
\item Music (diegetic or non-diegetic)
\end{itemize}

\section{Sound and silence}
Be wary of using silence (or black) as it is often perceived as though something has gone wrong. Better still to head towards the minute but keep something there.

\subsection{The Punch}
What we hear is what we haven't had time to see! \citep[]61{chion1990}

\subsection{Synchresis}
Chion's combination of \textit{syncrhonism} and \textit{synthesis}, the join between sound and image. Gestalt theories come into play here. 

\subsection{Space}
See \textit{Sonic Art: Recipes and Reasonings} for a discussion of space.  

\subsection{Time}
Sound that precedes action, sound that sums up action. Sound that raises questions.
Chion calls sound that precedes action \textit{active offscreen sound} and sound that envelops and stabilizes a scene \textit{passive offscreen sound}. Passive offscreen sound is the perfect extension of the scene and provides added value.
So you should ask yourself, `what is on screen and what is off screen?'

\subsection{Phonogeny}
Phonogeny is Chion's means to understand the quality of a sound \citep[101]{chion1990} in relation to the medium. (Note Chion's analogy of the not so beautiful women but `incredibly photogenic'). The same can be true for audio. However, current practice deprives us of the time to see and hear. This quote is worth repeating in full:

\begin{quotation}
This leads us to wonder what the disappearance of the notion of phonogeny is the symptom of. Perhaps it signals an important mutation, to our total everyday immersion in \textit{mediated acoustical reality} (sound is relayed by amplifiers and loudspeakers). The new sound reality has no difficulty supplanting unmediated acoustical reality in strength, presence, and impact, and bit by bit it is becoming the standard form of listening. It's a form of listening that is no longer perceived as a reproduction, as an image (with all this usually implies in terms of loss and distortion of reality), but as a more direct and immediate contact with the event. When an image has more presence than reality it tends to substitute for it, even as it denies its status of image.

\raggedleft{\citep[103]{chion1990}}
\end{quotation}

We now expect the hyper-real. 


