
\chapter{Syntax}
\label{syntax}

\section{Michel Chion's terminology}
\begin{itemize}
\item Added Value: sound that 'naturalises' the image
\item Anempathetic sound. Sound (usually diegetic music) that plays against the plot of the film. Chion cites a scene where a radio continues to play even when the character that turned it on has died.
\item Diegetic sound is source-visible or implied by the action of the film. It can be on-screen or off-screen.
\item Non-diegetic sound is not source-visible and is usually the musical score or additional foley/sfx to the implied (diegetic) sounds.
\item Empathetic sound is usually music that is completely within the mood of the action.
\item Reduced listening: qualitative listening independent of source or meaning (also partly quantitative). Spectral listening.
\item Semantic listening: communicative content (normally language).
\item Causal listening: listing out for source or cause (normally associated with the sound of what is seen but often used to pre-figure what is about to appear). 
\item Synchresis. Relationship between audio and visual content.
\end{itemize}

\subsection{Michel Chion \textit{Audio Vision} with examples}
\begin{itemize}
\item Eg 1. Bregman \textit{Persona, 1966}: Very complex sonically (and visually). Chion suggests watching the prologue with / without sound. 
\item Eg 2. Jacques Tati \textit{Mr. Hulot's Holiday}: Sonically very descriptive. 
\item Sound affecting the nature of time in a moving image. Sound may animate, it may accelerate, it may decimate. Chion suggests sound may \textit{vectorize} shots, directing time towards a goal.
\item Sound unifies the flow of images. It `bridges the visual breaks' \citep[p.47]{chion1990} 

\end{itemize}

\subsection{Post Chion: Martin Stig Andersen and \textit{Rocketman(2007)}}
\begin{itemize}
\item \url{http://www.martinstigandersen.dk/node/16}
\item \url{http://econtact.ca/12_4/andersen_audiovisual.html}
\end{itemize}

\section{Understanding aspects of footage}
Basic shot types include
\begin{itemize}
\item Extreme close-up (ECU)
\item Big close-up (BCU)
\item Close-up (CU)
\item Medium close-up (MCU)
\item Medium shot (MS)
\item Medium long shot (MLS)
\item Long shot (LS)
\item Very long shot (VLS)
\item Extreme long shot (ELS)
\item Two shot (2S): two people
\item Over the shoulder (OLS)
\end{itemize}

Complex shots may involve camera movement, lens adjustment as well as tracking the subject. A complex shot may pan, tilt, zoom, focus. 

