
\chapter{Syntax}
\label{syntax}

Michel Chion's terminology.
\begin{itemize}
\item Added Value: sound that 'naturalises' the image
\item Anempathetic sound. Sound (usually diegetic music)that plays against the plot of the film. Chion cites a scene where a radio continues to play even when the character that turned it on has died.
\item Diegetic sound is source-visible or implied by the action of the film. It can be on-screen or off-screen.
\item Non-diegetic sound is not source-visible and is usually the musical score or additional foley/sfx to the implied (diegetic) sounds.
\item Empathetic sound is usually music that is completely within the mood of the action.
\item Reduced listening: qualitative listening independent of source or meaning (also partly quantitative). Spectral listening.
\item Semantic listening: communicative content (normally language).
\item Synchresis. Relationship between audio and visual content.
\end{itemize}