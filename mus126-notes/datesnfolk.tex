\chapter{Dates and Times in history}
\label{datesnfolk}

\section*{A short list of dates, times and places that are of importance.}

%\begin{description}
%  \item[First] The first item
%  \item[Second] The second item
%  \item[Third] The third etc \ldots
%\end{description}


\begin{description}
\item [Charlemagne] In lands ruled by the Popes and the Franks, Charles - later to be called Chalemagne - had control over a vast kingdom taking in most of what is now France, Germany and Italy on/around 775. Chalemagne encouraged the pursuit of learning and people prospered. War was never far away however with the Vikings and the Magyars attacking in the late 10th Century. 
\item [Habsburg] Under Empirial rule, more local government was taken care of by hugely powerful families - dynasties, the largest of which was the Habsburgs. The first Habsburg `Emperor of Rome' was Rudolf IV 1272/1273. The Habsburg empire survived until 1806 when the title was abolished. 
\item [1066] The Normans, led by William landed on September 28, 1066 and ultimately defeated the king, Harold in the battle of Hastings. Because of familial alliances (or more probably their breakdown), William challenged Harold's claim to the throne. 
\item [The power of the church] The Crusades began in 1095 and started two centuries of violence between Christians and Muslims. 
\item [Law and Order] 1215 and King John signed Magna Carta, promising to abide by its laws. 
\item [The Black Death] Rather like Bird flu, the plague started somewhere far, far away but eventually reached every corner of the land. It reached London in 1349. With people tired and weak after war, it is hardly surprising they were susceptible. 
\item [The Hundred Years' War] Anglo-French relations were relatively stable due to the marital ties of Henry Plantagenet and Eleanor of Aquitaine. However, this was to break down when Edward III went to war against France in the defence of Aquitaine in 1357. 
\item [The Renaissance] Leonardo de Vinci (1452-1519), Michelangelo (1475-1564) Raphael (1483-1520) 
\item [Brave new world] Christopher Columbus sets sail in 1492. 
\item [Reformation] Martin Luther posts his 95 propositions in 1517 and the Protestant Catholic revolution had begun. 
\item [Elizabeth I] England and Spain had been allies through marriage but when Elizabeth succeeded the throne from Queen Mary (Catholic) in 1558, a division appeared. With Spain and Portugal now in league, war was to follow. Francis Drake was one of the heros of the war. 
\item [Shakespeare] With a gift for capturing the essence of the age Shakespeare (1564, 1616) was an acclaimed writer in his time.
\item [The Thirty Years' War] Bitter rivalries between the Holy Roman Empire (dominated by the Habsburgs) and surrounding nations, notably France. Catholic fought Catholic (France vs. Roman Empire using Spanish armies). 
\item [Roundheads and Cavaliers] Death of Charles I 1649. English civil war middle of the 17th Century. 
\item [The Sun King] Louis XIV as a puppet of the Church, attempts absolute rule. Future kings Louis XV and XVI gradually see their empire tumble into revolution. 
\item [The British in India] The British East India Company - something that just got out of hand. Read about the `black hole of Calcutta' (1756).
\item [Don't forget the East] Frederick the Great ruled over Prussia for over 40 years. 
\item [Canada] Britain vs. France batte for Canada in 1759
\item [Australia] Discovered by Cook in 1770 though seen by many before that time. 
\item [Independence] Benjamin Franklin, Thomas Jefferson and John Adams in 1776
\item [Revolution] The storming of the Bastille in 1789. The rise, and ultimately the fall (1794) of `citizen Robespierre'. 
\item [Napoleon] leads his overstretched army in to Russia and to a wintry death in 1812
\item [Communism] Karl Marx (1818-1883)
\item [Americans in Japan] After 1853, American ships sail into Yokohama and Nagasaki to trade with Japan. 
\item [Civil War in America] Abraham Lincoln (1809-1865) proclamation in 1862 to end slavery. 
\item [Russian Revolution] 1905 Tsar Nicholas II quashed a large uprising - Bloody Sunday. 
\item [First World War] Following the assassination of Archduke Franz Ferdinand by Serbian `Black Hand' Austria using Germany as ally decided to neutralise Serbia. Russia came to Serbia's defence. Although France was neutral they could not avoid being caught up in the war as Germany declared war on Russian and invaded Luxembourg prior to invading Belgium. Germany declared war on France. Britain came to Belguim's aid when the German's invaded. 1914, Britain and France were at war. 
\item [Battles] of Britain and WWII. 
\item [Other continuing battles with a history] Jews and Palestinians. Key moments include independence of Israel in 1948. Pursuant to this and the division of Palestine, a continuance of Arab-Israeli conflicts that continue to this day.    

\end{description}



