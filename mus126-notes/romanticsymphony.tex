
\chapter{The Romantic Symphony}
\label{romanticsymphony}

%\section{Scores}

\section{Brahms Symphony No 1 }
\begin{itemize}
\item Listen: \url{https://www.youtube.com/watch?v=o3a4v1TWUNo}
\item Score: \url{http://imslp.org/wiki/Symphony_No.1,_Op.68_%28Brahms,_Johannes%29}
\item Reading: \url{http://www.jstor.org/stable/pdfplus/20696500.pdf}
\end{itemize}

Symphony No.1 in C minor Op.68. Completed in 1876 but started in 1855 so interesting to probe the gestation period. 

Johannes Brahms (1833-1897, born in Hamburg). An exceptional body of work including works for piano, chamber ensemble, works for the female voice and four symphonies. Brahms was an excellent pianist and loved to be in love. His muse was the Clara Schumann. Brahms knew the Schumanns through their mutual friend and violin virtuoso Joseph Joachim. Unfortunately Robert Schumann was nearing the end of his life. He tried to commit suicide in 1854 and died in an asylum in 1856. Tragically sad end to the romantic composer's life. Brahms was superb at Lieder form but equally competent at the string quartet. With larger scale works, the violin concerto remains one of the key romantic concertos. 

Section{Europe in the middle 19th Century}
Europe was naturally going through political turmoil during the middle of the century. There were revolutions in France, Italy, Hungary and Germany. Technology was moving apace. Although the first known photograph was taken in 1827, it was not until the later part of the century that George Eastman created a method that was cheap enough to be mass-marketed (Kodak). Meanwhile Michael Faraday was making the first experiments generating electricity by passing magnets through coils of wire. By the end of the century street lighting was electric and the light bulb had been developed. 

Prior to Brahms' birth notable artistic endeavours included:
\begin{itemize}
\item Jane Austin: \textit{Pride and Prejudice} (1813)
\item Mary Shelly: \textit{\href{http://www.gutenberg.org/files/84/84-h/84-h.htm}{Frankenstein}} (1818)
\item Carl Maria von Weber: Der Freisch\"utz (1821)
\item John Constable: \textit{\href{http://www.nationalgallery.org.uk/paintings/john-constable-the-hay-wain}{The Hay Wain}} (1821)  
\end{itemize}

Der Freisch\"utz is perhaps best known because of the `Wolf's Glen' scene where magic bullets are cast and the devil is conjured. It is here that romantic ideas of the supernatural appear on stage and in music. In this respect the Hay Wain is both classical and romantic. The way nature was brought to life and the degree of subjective interpretation began to move painting from the figurative to the subjective. Alongside Constable we have the landscape painting of Joseph Turner (1775-1851). You can see the great attention to detail and figure but at the same time, an approach to colour that is bordering on the impressionist. 

Edgar Allen Poe introduced ideas of the supernatural and the macabre in his key texts of the mid 19th Century including `The Fall of the House of Usher', `The Pit and the Pendulum', and poems such as `\href{https://www.youtube.com/watch?v=0K6-wO94-6I}{The Raven}'.  

Going back further still, the American declaration of independence of 1776 (read about the Boston Tea Party) led others to believe in emancipation from dictators. In particular the French, who revolted against Louis XVI and Marie Antoinette in 1789 by storming the Bastille. 

And in 1812 Napoleon invaded Russia but this was the beginning of the end for him. The gruesome nature of war is captured in paintings such as Goya's `The Third of May 1808', during the Peninsular war between France and Spain. We see the French slaughtering helpless Spanish civilians.   

And of course in 1824 we have Beethoven's Choral Symphony, stretching musical form and forces to the maximum. 

The turmoil the romantic painting (see Turner's Snow Storm: Steam-Boat off a Harbour's Mouth of 1842) can be heard in music of Wagner in the middle of the century (Operas depicting salvation through death - Der fleiglende Holl\"ander, Tristan und Isolde for example) and in the writing of the Bront\"e sisters (Emily Bront\"e's \href{http://www.wuthering-heights.co.uk/index.php}{Wuthering Heights} of 1847 (\href{http://www.gutenberg.org/ebooks/768}{read it here}).

Verdi (1813-1901) was using his operas to investigate character - in particular their flaws and failings, famously drawing upon Shakespeare (for Macbeth(1865), and towards the end of the century, Otello(1887) and Falstaff(1893)). 
At the time Brahms was starting to write the First Symphony, Verdi had achieved fame with three operas: Rigoletto(1851), Il Trovatore(1853) and La Traviata(1853). But although Verdi might sound like Wagner, he did not know the man's music. Rigoletto is particularly tragic but endears itself through highly memorable arias such as `La donne \'e mobile'. Following hot on the heels of Verdi was Giacomo Puccini (1858-1924). Puccini heard a performance of Verdi's \textit{A\"ida} and this convinced him that the family business of organist was not for him. Puccini's first opera \textit{Manon Lescaut} (1893) was extremely successful and he is known as an opera composer. We will look at \textit{La Boh\`eme} (1896) - another masterpiece. 

Wagner's Ring cycle had come to completion and the Festspielhaus was to premiere \textit{Das Rheingold} on 13th August 1876. Brahms' First Symphony premiered on November 4, 1786. 

\subsection{Brahms Symphony No. 1: details} 


\section{Bruckner Symphony No 4 (1874)}
\begin{itemize}
\item Listen: \url{https://www.youtube.com/watch?v=gcBg-tXn0fs}
\item Score: \url{http://imslp.org/wiki/Symphony_No.4_in_E-flat_major,_WAB_104_%28Bruckner,_Anton%29}
\end{itemize}

Completely contemporary to Brahms is Bruckner (1824-1896) yet their music could not be more different. Both had romantic vision tempered by classical contraints. Bruckner followed in the Wagnerian tradition of the grand statement and this is easily seen and heard in his symphonies. As with Puccini's life-changing audition of \textit{A\"ida} so it was the premiere of Wagner's \textit{Tristan und Isolde} (composed between 1857 and 1859) in 1865 that convinced Bruckner that the life of a composer (not a teacher) was for him. 

Prior to Bruckner's fourth symphony the everyman's life is continually racked by war particularly in and around Austria. Britain and Russia had been at war in the Crimea from 1854-6. On the other side of the Atlantic and in another world emigrant Jew, Levi Strauss began making Jeans for coal miners, opening his business in 1853 and producing Jeans in 1871.  

The 1860s saw 
\begin{itemize}
\item \`Edouard Manet paint \textit{\href{https://www.khanacademy.org/humanities/becoming-modern/avant-garde-france/realism/v/manet-le-d-jeuner-sur-l-herbe-luncheon-on-the-grass-1863}{D\'ejeuner sur l'Herbe}} (1863).
\item Lewis Carroll write \textit{\href{http://www.gutenberg.org/files/11/11-h/11-h.htm}{Alice's Adventures in Wonderland}} (1865)
\item Leo Tolstoy write \textit{War and Peace} (1869)
\end{itemize}

The American Civil War had fractured north and south but in Italy a country hugely divided, occupied and dominated began to coalesce. Italy's operatic composers helped enormously by rousing public patriotism.

The 1870s, the time of Symphony 4 saw the rise of impressionism in painting. Claude Monet, Edgar Degas, Pierre-August Renoir exhibited their work, one of which was Monet's \textit{Impression: Sunrise} of 1873. 
And shortly following this, in 1894, D\'ebussy writes \textit{Prelude a l'apr\`es-midi d'un Faune} (and remember that La Boheme is just two years after this). 

\subsection{Bruckner Symphony No. 4: details} 













