
\chapter{The Romantic Symphony}
\label{romanticsymphony}

%\section{Scores}

\section{Brahms Symphony No 1 }
\begin{itemize}
\item Listen: \url{https://www.youtube.com/watch?v=o3a4v1TWUNo}
\item Score: \url{http://imslp.org/wiki/Symphony_No.1,_Op.68_%28Brahms,_Johannes%29}
\item Reading: \url{http://www.jstor.org/stable/pdfplus/20696500.pdf}
\end{itemize}

Symphony No.1 in C minor Op.68. Completed in 1876 but started in 1855 so interesting to probe the gestation period. 

Johannes Brahms (1833-1897, born in Hamburg). An exceptional body of work including works for piano, chamber ensemble, works for the female voice and four symphonies. Brahms was an excellent pianist and loved to be in love. His muse was Clara Schumann, the wife of Robert Schumman. Brahms knew the Schumanns through their mutual friend and violin virtuoso Joseph Joachim. Robert Schumann lived a troubled and short life. He tried to commit suicide in 1854 and died in an asylum in 1856. Tragically sad end to the romantic composer's life. Brahms was superb at Lieder form but equally at home with all manner of ensembles: string quartet op.67, piano trios, string sextets. With larger scale works, the violin concerto remains one of the key romantic concertos. 

\section{Europe in the middle 19th Century}
Europe was naturally going through political turmoil during the middle of the 19th century. There were revolutions in France, Italy, Hungary and Germany. Technology was moving apace. Although the first known photograph was taken in 1827, it was not until the later part of the century that George Eastman created a method that was cheap enough to be mass-marketed (Kodak). Meanwhile Michael Faraday was making the first experiments generating electricity by passing magnets through coils of wire. By the end of the century Thomas Edison had demonstrated a commercially affordable light bulb and most street lighting was electric. 

Prior to Brahms' birth notable artistic endeavours included:
\begin{itemize}
\item Jane Austin: \textit{Pride and Prejudice} (1813)
\item Mary Shelly: \textit{\href{http://www.gutenberg.org/files/84/84-h/84-h.htm}{Frankenstein}} (1818)
\item Carl Maria von Weber: \textit{Der Freisch\"utz} (1821)
\item John Constable: \textit{\href{http://www.nationalgallery.org.uk/paintings/john-constable-the-hay-wain}{The Hay Wain}} (1821)  
\end{itemize}

Der Freisch\"utz is perhaps best known because of the `Wolf's Glen' scene where magic bullets are cast and the devil is conjured. It is here that romantic notions of the supernatural appear on stage and in music. Constable's \textit{The Hay Wain} is both classical and romantic. The way nature was brought to life and the degree of subjective interpretation began to move painting from the figurative to the subjective. Alongside Constable we have the landscape painting of Joseph Turner (1775-1851). You can see the great attention to detail and figure but at the same time, an approach to colour that is bordering on the impressionist. 

Edgar Allen Poe introduced ideas of the supernatural and the macabre in his key texts of the mid 19th Century including `The Fall of the House of Usher', `The Pit and the Pendulum', and poems such as `\href{https://www.youtube.com/watch?v=0K6-wO94-6I}{The Raven}'.  

%Going back further still, the American declaration of independence of 1776 (read about the Boston Tea 
%Party) led others to believe in emancipation from dictators. In particular the French, who revolted
%against Louis XVI and Marie Antoinette in 1789 by storming the Bastille. 

In 1812 Napoleon invaded Russia but this was the beginning of the end for him. The gruesome nature of war is captured in paintings such as Goya's `The Third of May 1808', during the Peninsular war between France and Spain. We see the French slaughtering helpless Spanish civilians.   

And of course in 1824 we have Beethoven's Choral Symphony, stretching musical form and forces to the maximum. 

The emotional turmoil of much romantic painting (see Turner's Snow Storm: Steam-Boat off a Harbour's Mouth of 1842) can be heard in music of Wagner in the middle of the century (Operas depicting salvation through death - Der Fleiglende Holl\"ander, Tristan und Isolde for example) and in the writing of the Bront\"e sisters (Emily Bront\"e's \href{http://www.wuthering-heights.co.uk/index.php}{Wuthering Heights} of 1847 (\href{http://www.gutenberg.org/ebooks/768}{read it here}).

Verdi (1813-1901) was using his operas to investigate character - in particular their flaws and failings, famously drawing upon Shakespeare (for Macbeth(1865), and towards the end of the century, Otello(1887) and Falstaff(1893)). 
At the time Brahms was starting to write the First Symphony, Verdi had achieved fame with three operas: Rigoletto(1851), Il Trovatore(1853) and La Traviata(1853). But although Verdi might sound like Wagner, he did not know the man's music. Following hot on the heels of Verdi was Giacomo Puccini (1858-1924). Puccini heard a performance of Verdi's \textit{A\"ida} and this convinced him that the family business of church organist was not for him. Puccini's first opera \textit{Manon Lescaut} (1893) was extremely successful and he is mainly known as an opera composer. We will look at \textit{La Boh\`eme} (1896) - another masterpiece. 

Wagner's Ring cycle had come to completion and the Festspielhaus was to premiere \textit{Das Rheingold} on 13th August 1876. Brahms' First Symphony premiered on November 4, 1876. 

\section{Brahms Symphony No. 1: details} 

\subsection{Viewing}
\href{http://bobnational.net/record/92680/media_id/99760}{Box of Broadcasts: The symphony (BBC4) 3/4.} 

\subsection{First movement}

The movement begins with a slow introduction that highlights some of the key cells of the piece and focuses attention upon the strings and oboe, important declamatory instruments for the first movement. The opening introduction is marked \textit{Un poco sostenuto} and \textit{pesante} but this is interpreted wildly from conductor to conductor (A very brief search shows Andrew Manze's recording taking a brisk 2:16 with Mariss Jansons taking a ponderous 2:55 - and 40 seconds makes all the difference here) the movement picks up the pace for a spirited allegro. The strings present the theme in C minor.

\begin{figure}[H]
\centering
\includegraphics[scale=0.2]{brahms1mvt1a}\caption{Brahms Symphony No.1 mvt1 First subject}
\label{fig:b1m1first}
\end{figure}

At figure C a transition begins with lilting rhythms (important for driving the music forward during the development section). The movement slows and the oboe states the second theme.

\begin{figure}[H]
\centering
\includegraphics[scale=0.2]{brahms1mvt1b}\caption{Brahms Symphony No.1 mvt1 Second subject}
\label{fig:b1m1second}
\end{figure}

But it is not long before the pace picks up again with heavily articulated strings.

\begin{figure}[H]
\centering
\includegraphics[scale=0.2]{brahms1mvt1c}\caption{Brahms Symphony No.1 mvt1 articulation 1}
\label{fig:b1m1rhythmic1}
\end{figure}

The exposition is not normally repeated (although Manze's recording does repeat). The move to B major is astonishing. 

Brahms makes a clever move as he develops the articulated theme from figure~\ref{fig:b1m1rhythmic1} towards something that mixes articulated and sustained (figure~\ref{fig:b1m1rhythmic2}).

\begin{figure}[H]
\centering
\includegraphics[scale=0.2]{brahms1mvt1d}\caption{Brahms Symphony No.1 mvt1 articulation 2}
\label{fig:b1m1rhythmic2}
\end{figure}

A huge sequential passage (bar 321) heralds the recapitulation in the home key and a major second subject return. 

After a heavily punctuated section alternating wind and strings, the coda seals the movement calmly but beguilingly contrasting major and minor modes leaving perhaps a sense of unease at the end.  

\subsection{Second movement} 
The second movement is an altogether stately affair. The oboe and strings again share the key themes but there is less thematic unity than in the previous movement. The slow solo violin heralds the end of the movement. 

There are some very interesting chromatic passages at the close of this movement that suddenly remind us of the opening of the symphony. 

\begin{figure}[H]
\centering
\includegraphics[scale=0.2]{brahms1mvt2chromatic}\caption{Brahms Symphony No.1 mvt2 bar 116}
\label{fig:b1m2chromatic}
\end{figure}

with...

\begin{figure}[H]
\centering
\includegraphics[scale=0.2]{brahms1mvt1chromatic}\caption{Brahms Symphony No.1 mvt1 Allegro woowind bar 38}
\label{fig:b1m2chromatic}
\end{figure}

\subsection{Third movement}
The third movement is therefore much lighter in design and is in A$\flat$ major.

\subsection{Fourth movement}
The final movement opens with a slow introduction. Brahms experiments here with quick conversational writing especially on strings (pizzicato accelerations and rapid exchanges of notes - which if the strings are placed opposite each other rather than next to each other can produce interesting results). The \textit{Piu Andante} at bar 30 begins another series of stately passages in C major including a beautiful brass chorale. This leads to the main romantic theme of the movement (\textit{Allegro non troppo, ma con brio}) which is simple and bold in design with a I/V harmonisation (and surely with a nod towards Beethoven's 9th).  
%%%%%%%%%%%%%%%%%%%%%%%%%%%%%%%%%%%%%%%%%%%%%%%%%


\section{Bruckner Symphony No 4 (1874)}
\begin{itemize}
\item Listen: \url{https://www.youtube.com/watch?v=gcBg-tXn0fs}
\item Score: \url{http://imslp.org/wiki/Symphony_No.4_in_E-flat_major,_WAB_104_%28Bruckner,_Anton%29}
\end{itemize}

Completely contemporary to Brahms is Bruckner (1824-1896) yet their music could not be more different. Both had romantic vision tempered by classical constraints. Bruckner followed in the Wagnerian tradition of the grand statement and this is easily seen and heard in his symphonies. As with Puccini's life-changing audition of \textit{A\"ida} so it was the premiere of Wagner's \textit{Tristan und Isolde} (composed between 1857 and 1859) in 1865 that convinced Bruckner that the life of a composer (not a teacher) was for him. 

Prior to Bruckner's fourth symphony daily life is continually racked by war particularly in and around Austria. Britain and Russia had been at war in the Crimea from 1854-6. \footnote{On the other side of the Atlantic; in another world and on the lighter side of history, emigrant Jew, Levi Strauss opened his business in 1853 and began producing Jeans in 1871.}  

The 1860s saw 
\begin{itemize}
\item \`Edouard Manet paint \textit{\href{https://www.khanacademy.org/humanities/becoming-modern/avant-garde-france/realism/v/manet-le-d-jeuner-sur-l-herbe-luncheon-on-the-grass-1863}{D\'ejeuner sur l'Herbe}} (1863).
\item Lewis Carroll write \textit{\href{http://www.gutenberg.org/files/11/11-h/11-h.htm}{Alice's Adventures in Wonderland}} (1865)
\item Leo Tolstoy write \textit{War and Peace} (1869)
\end{itemize}

The American Civil War had fractured north and south but in Italy a country hugely divided, occupied and dominated began to coalesce. Italy's operatic composers helped enormously by rousing public patriotism.

The 1870s, the time of Symphony 4 saw the rise of impressionism in painting. Claude Monet, Edgar Degas, Pierre-August Renoir exhibited their work, one of which was Monet's \textit{Impression: Sunrise} of 1873. 
And shortly following this, in 1894, D\'ebussy writes \textit{Prelude a l'apr\`es-midi d'un Faune} (and remember that La Boheme is just two years after this). 

\subsection{Bruckner Symphony No. 4 in E$\flat$ major: details} 

One of Bruckner's most popular symphonies. The first movement was worked on while Bruckner finished his 3rd symphony with the first movement being completed on January 2nd, 1874. The second movement came in the spring of that year; the Scherzo in the summer. The Finale took just one month to complete with the finishing touches to the work being completed by August 31st. The orchestration of the work was completed in November. 

The revision history of the work is worthy of further analysis, indeed it has become known as the `Bruckner problem'. The version of the symphony generally performed today comes from 1876-1880 with the biggest difference from the first version being a new Scherzo and a replacement Finale. The symphony has no religious programme (Bruckner wrote a significant amount of sacred music, being a devout Catholic and organist) and moves between romantic programme music and absolute music. It is `Romantic' in the Herculean sense: literally and metaphorically huge. The programme of the work includes reference to the breaking of day, hunting music, entertainment music; indeed even birdsong is referenced. 

\begin{quotation}
A medieval town - dawn of morning - the morning calls are sounded from the towers of the city - the gates of the town are opened - on fine horses the brilliant knights ride out into the open - the forest with its beauties receives them - forest murmurs - singing birds - a fine romantic picture. 
\end{quotation}

The Scherzo originally contained the inscription `Dance strain during a repast at hunting.' This is a  symphony representing nature along the lines of Beethoven's \textit{Pastoral} symphony. For a full pairing down of the symphony, you are invited to read \href{https://urresearch.rochester.edu/institutionalPublicationPublicView.action?institutionalItemId=5839}{Joseph Mark Lalumia's 1978 thesis} on the symphony. 

The bold vision of the work is suggested by the open fifths in the opening bars. 

\begin{figure}[H]
\centering
\includegraphics[scale=0.2]{bruckner4a}\caption{Bruckner Symphony No.4 first movement opening call}
\label{fig:bruckner1a}
\end{figure}

The orchestration is majestic; the thematic writing conversational, with relatively simple themes toggling back and forth between woodwind, brass and strings. 

%theme at C in first movement with counter theme%

A 2 quarter-note, triplet quarter-note theme with cross-wise motion is also extremely declamatory.

\begin{figure}[H]
\centering
\includegraphics[scale=0.2]{bruckner4b}\caption{Bruckner Symphony No.4 mvt1 declamatory theme}
\label{fig:bruckner1b}
\end{figure}

But there are some very lyrical moments where Bruckner relaxes the mood somewhat. 

\begin{figure}[H]
\centering
\includegraphics[scale=0.2]{bruckner4c}\caption{Bruckner Symphony No.4 mvt1 lyrical counterpoint}
\label{fig:bruckner1c}
\end{figure}

Bruckner takes great liberties with unison and `at the octave' melodic lines. 
Sections contrast and wrestle, one against each other like granite slabs (note the changes before figure G in the first movement). G itself is quite operatic in its suggestiveness. The thematic play is block against block with transposition and orchestration often the only developments over time. (Compare the way Debussy handles thematic material in his work \textit{Jeux} - to be analysed later -  where themes are intricately similar but different). Despite the relative difficulty in creating a hierachy of thematic material in the exposition, the movement is in sonata form (with development at bar 193, recapitulation at bar 365, and a coda at bar 501). 

\subsection{Second movement}

The second movement is also in sonata form. The movement begins in C minor with the cellos playing the main theme. Just as in the quieter moments of the first movement the timpani play a huge role in maintaining the background. In places, the brass harmonies hint at Wagner. Finally around figure M, Bruckner decorates his harmonies with more intense string writing. However, the movement always seems to be hanging on the edge; frail, despite the huge horn calls. 

The two main themes from this movement are: 
\begin{figure}[H]
\centering
\includegraphics[scale=0.2]{bruckner4d}\caption{Bruckner Symphony No.4 mvt2 opening string theme}
\label{fig:bruckner1c}
\end{figure}

and

\begin{figure}[H]
\centering
\includegraphics[scale=0.2]{bruckner4e}\caption{Bruckner Symphony No.4 mvt2 lyrical theme}
\label{fig:bruckner1c}
\end{figure}

\subsection{Third movement}

This movement clearly wares its hunting call on its sleeve. Typical of Bruckner too that we arrive at one glorious crescendo moment and the next passage is back in the meadows; light and airy. All the thematic material comprises larger intervals (4ths and 5ths). As tension builds by sequence, Bruckner uses more pedal points in this movement. The trio has a very Mahler-like feel to it. Clearly they both knew each other's work. Whilst Mahler started writing symphonies in 1888, working voraciously averaging one symphony every two years, Bruckner's themes and bravura style must have been appreciated by his colleague; so too his wunderhorn-like melodies. (Compare Mahler's first symphony with its L\"andler and themes from \textit{Lieder eines fahrenden Gesellen}.) This mode of pastoral description is common to both composers.  

\subsection{Fourth movement}

Picking up where the third movement left off, on a B$\flat$, the fourth movement seems almost like a continuation of the third. Back has come the 2-quarter 3-triplet-quarter hammer motif from the first movement. Naturally the B$\flat$ is the dominant of E$\flat$ at bar 79. Quickly we are back to something much more cheeky which modulates aggressively through the keys. Figure F again uses tonic/ submediant harmonies.  The introduction recurs at 203 and again leads to a large climax but this movement has perhaps spent itself somewhere before the close such is its tumultuous and never-ending shift of pace. 

Bruckner shares material amongst the movements which gives the work an overall homogeneous feel (although at well over an hour to play, perhaps a little tiresome). 


