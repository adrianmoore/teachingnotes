
\chapter{Opera}
\label{opera}


\section{Romanticism and Puccini: La Boh\`eme}
\begin{itemize}
\item Listen: \url{https://www.youtube.com/watch?v=ntg9vXxAia8}
\item Score: \url{http://imslp.org/wiki/La_boh%C3%A8me_%28Puccini,_Giacomo%29}
\item Reading (not for comment): \url{http://www.pittsburghopera.org/files/file/Study%20Guide%20for%20La%20boheme.pdf}
\end{itemize}

Giacomo Puccini was born into a musical family. Puccini's father was an organist and teacher. Puccini was the fifth of eight children, five girls, three boys. As Puccini learned his trade, his composition teacher introduced him to Verdi's scores (\textit{Rigoletto}, \textit{La traviata}). And from hearing \textit{A\"ida} he began to compose less for the organ. Puccini continued his studies in Milan under the tutelage of Antonio Bazzini beginning in 1880, also with Amilcare Ponchielli. He was joined in Milan by Pietro Mascagni, composer of Cavalleria rusticana (1890), an opera that started the \textit{Verismo} tradition (emphasis upon realism). In 1883 Puccini composed a ten minute orchestral piece, \textit{Capriccio Sinfonico}. Already operatic in nature; dramatic and poignant, one  can hear the beginnings of his phrase structure, the mood swings and the swirling, haunting melodies. The whole work sounds like an introduction and in a way, it is perhaps an overture to larger things. And then all of a sudden we hear the opening of \textit{La Boheme}. 

Just as `getting signed' today is key to success, so it was back then and the two main publishers were Ricordi and Lucca. After two less well known pieces \textit{Le villi} and \textit{Edgar} with libretti by Fernando Fontana, Ricordi commissioned new work and Puccini went through librettist after librettist. It seems Puccini was very picky about who he worked with eventually siding with Giuseppe Giacosa and Luigi Illica. This opera was \textit{Manon Lescaut}. It is interesting to note that Illica and Giacosa were to remain faithful to Puccini for future successful operas. \textit{Manon Lescaut} was completed in 1892 after three years hard work. Immediately it has signature tenor lines and languorous phrasing. Recitative and Aria have disappeared as has a post-Verdi accompanimental style. 

The success of \textit{Manon Lescaut} now set Puccini fair financially. 

\textit{La boh\`eme} comes from Henri M\"urger's Sc\`enes de la vie de boh\`eme. Bohemianism is associated with carefree, gypsy-like lifestyle. There is a relatively intense history pertaining to the emergence of Puccini's version. It seems composers and publishers were all scrambling to the same story. Moreover, the libretto's gestation was lengthy and revised many times. Once the libretto was in good shape, Puccini worked on the score: 1895 saw the completion of the music. The premiere was set for Teatro Regiio, Turin; date 1st February 1896. Despite recommendations for conductors from Puccini, Ricodi selected young conductor Auturo Toscanini.

To top it all, \textit{La boh\`eme} received mixed reviews. 

\subsection{Synopsis}
The Opera is set in the Latin Quarter of Paris around 1830. Four artists are living a poor life. Rodolfo (poet), Marcello (painter), Schaunard (musician) and Colline (philosopher). They are ``Bohemians'', penniless but passionate about life. It's Christmas Eve and three of the men leave the flat to go to the cafe, leaving Rodolfo behind. Enter Mimi, frail, delicate (and ill) to ask for a candle. Love lights up the room. They eventually join the group at the cafe along with Musetta who, accompanied by her much older admirer Alcindoro tempts her ex, Marcello to fall in love with Musetta again. 

A month later. Mimi finds Marcello to tell him that Rudolfo, in a fit of jealousy has left her. When Marcello questions Rodolofo, he explains his departure was due to the fact he can not support her as she is dying (of consumption - TB). Mimi overhears this conversation but they meet and love, once again endures. Marcello and Musetta - now living together - also argue. 

In the fourth act, Marcello and Rudolfo are lamenting love. They are both separated from their loved ones. Schaunard and Colline try to brighten the atmosphere but the merriment is arrested when Musetta arrives with a dying Mimi. The friends go off to search for a doctor leaving Rudolfo and Mimi together. 

\subsection{What makes La Boh\`eme ground-breaking?}
Look at a review from critic of the time Eduard Hanslick (1825-1904)
\begin{quotation}
The few earlier operas that deal seriously with affairs between wanton courtesans and weak youths (\textit{La Traviata, Camen}, and most recently \textit{Manon}) have at least dressed them in picturesque national or historic garb, or set them in romantic surroundings and thus raised them out of the lowest regions of everyday wretchedness. With \textit{La Boh\`eme} our composers take the last step towards the naked, prosaic dissoluteness of our time: heroes in loud-checked trousers, gaudy ties and crumpled felt hats, cigarette butts in their mouths, their companions in bonnets and scanty shawls. This is new, a sensational break with the last romantic and artistic traditions of opera.
\end{quotation}

There's a new, more visceral depiction of everyday life. But amidst the mundane there are moments of pure poetry. In particular note the difference between the words of Mimi when she says, `actually my name is Lucia'. 

\subsection{Musical details}

The opening bravura theme comes from the earlier \textit{Capriccio Sinfonico}. 

\begin{figure}[H]
\centering
\includegraphics[scale=0.2]{boheme1}\caption{La Boh\`eme opening moments}
\label{fig:boheme1}
\end{figure}

Rodolfo's theme is a plaintive call swaying through high notes.

\begin{figure}[H]
\centering
\includegraphics[scale=0.2]{rodolfotheme}\caption{Rodolfo's theme}
\label{fig:rodolfotheme}
\end{figure}
 
The themes are simple but poignant. Note Rodolfo's melody as he and Mimi search for the lost key. This tune could not be easier to remember but its falling (and subsequent rise) is a search but a forlorn one. 

\begin{figure}[H]
\centering
\includegraphics[scale=0.2]{rodolfokey}\caption{Rodolfo's key theme}
\label{fig:rodolfokey}
\end{figure}




