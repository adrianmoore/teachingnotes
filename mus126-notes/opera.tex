
\chapter{Opera}
\label{opera}


\section{Romanticism and Puccini: La Boh\`eme}
\begin{itemize}
\item Listen: \url{https://www.youtube.com/watch?v=ntg9vXxAia8}
\item Score: \url{http://imslp.org/wiki/La_boh%C3%A8me_%28Puccini,_Giacomo%29}
\item Reading (not for comment): \url{http://www.pittsburghopera.org/files/file/Study%20Guide%20for%20La%20boheme.pdf}
\end{itemize}

Giacomo Puccini was born into a musical family. Puccini's father was an organist and teacher. Puccini was the fifth of eight children, five girls, three boys. As Puccini learned his trade, his composition teacher introduced him to Verdi's scores (\textit{Rigoletto}, \textit{La traviata}). And from hearing \textit{A\"ida} he began to compose less for the organ. Puccini continued his studies in Milan under the tutelage of Antonio Bazzini beginning in 1880, also with Amilcare Ponchielli. He was joined in Milan by Pietro Mascagni, composer of Cavalleria rusticana (1890), an opera that started the \textit{Verismo} tradition (emphasis upon realism). In 1883 Puccini composed a ten minute orchestral piece, \textit{Capriccio Sinfonico}. Already operatic in nature; dramatic and poignant, one  can hear the beginnings of his phrase structure, the mood swings and the swirling, haunting melodies. The whole work sounds like an introduction and in a way, it is perhaps an overture to larger things. And then all of a sudden we hear the opening of \textit{La Boheme}. 

Just as `getting signed' today is key to success, so it was back then and the two main publishers were Ricordi and Lucca. After two less well known pieces \textit{Le villi} and \textit{Edgar} with libretti by Fernando Fontana, Ricordi commissioned new work and Puccini went through librettist after librettist. It seems Puccini was very picky about who he worked with eventually siding with Giuseppe Giacosa and Luigi Illica. This opera was \textit{Manon Lescaut}. It is interesting to note that Illica and Giacosa were to remain faithful to Puccini for future successful operas. \textit{Manon Lescaut} was completed in 1892 after three years hard work. Immediately it has signature tenor lines and languorous phrasing. Recitative and Aria have disappeared as has a post-Verdi accompanimental style. 

The success of \textit{Manon Lescaut} now set Puccini fair financially. 

La boheme comes from Henri M\"urger's Sc\`enes de la vie de boh\`eme. Bohemianism is associated with carefree, gypsy-like lifestyle. 
