
\begin{table}[h!]
\begin{tabular}{p{8.0cm}|p{8.0cm}} \hline

1. Moondrunk & 2. Colombine \\\hline
The wine which through the eyes we drink & The pallid buds of moonlight \\
Flows nightly from the moon in torrents, & Those pale and wondrous roses \\
And as a spring-tide overfloows & Bloom in the nights of summer- \\
The far and distant land. & O could I pluck but one! \\
Desires terrible and sweet & My heavy heart to lighten, \\
Unnumbered drift in floods abounding. & I search in darkling river \\
The wine which through the eyes we drink & The pallid buds of moonlight, \\
Flows nightly from the moon in torrents. & Those pale white wondrous roses. \\
The poet, in an ecstasy, & Fulfilled would be my longing \\
Drinks deeply from the holy chalice, & If I could softly gather, \\
To heaven lifts up his entranced & With gentle care besprinkle \\
Head, and reeling quaffs and drains down & Upon your dark brown tresses \\
The wine which through the eyes we drink. & The moonlight's pallid blossoms. \\\hline

3. The Dandy & 4. A Chlorotic Laundry Maid \\\hline
A phantasmagorial light ray & Washes nightly white silk garments; \\
Illumines tonight all the crystalline flasks & Naked, snow-white silvery foreams \\
On the holy, sacred, ebony wash-stand & Stretching downward to the flood. \\
Of the taciturn dandy of Bergamo & Through the glade steal gentle brezes. \\
In sonorous bronze-enwrought chalice & Through the glade steal gentle brezes. \\
Laughs brightly the fountain's metallic sound, & Softly playing o'er the stream. \\
A phantasmagorial light ray & A chlorotic laundry maid \\
Illumines tonight all the crystalline flasks. & Washes nightly white silk garments. \\
Pierrot with countenance waxen & And the gentle maid of heaven. \\
Stands musing and thinks & By the branches softly fondled. \\
How he tonight will paint. & Spreads on the dusky meadows \\
Rejecting the red and the green of the east & All her moonlight-bewoven linen \\
He bedaubs all his face in the latest of styles & A chlorotic laundry maid. \\
With a phantasmagorial moonbeam. & \\\hline


5. Valse de Chopin & 6. Madonna \\\hline
As a lingering drop of blood & Rise, O mother of all sorrows,\\
Stains the lip of a consumptive, & From the alter of my verses!\\
So this music is pervaded& Blood pours forth from thy lean bosom\\
By a morbid deathly charm.& Where the sword of frenzy pierced it.\\
Wild ecstatic harmonies & Thy forever gaping gashes\\
Disguise the icy touch of doom, & Are like eyelids, red and open.\\
As a lingering drop of blood & Rise, O mother of all sorrows,\\
Stains the lip of a consumptive. & From the alter of my verses.\\
Ardent, joyful, sweet and yearning, & In the lacerated arms\\
Melancholic sombre waltzes, & Holdst thou thy Son's holy body,\\
Coursing ever through my senses & Manifesting Him to mankind-\\
Like a lingering drop of blood! & Yet the eyes of men avert themselves,\\
&  O mother of all sorrows!\\\hline
\end{tabular}
\end{table}


\newpage
\begin{table}[h!]
\begin{tabular}{p{8.0cm}|p{8.0cm}} \hline

7. The Ailing Moon & 8. Night \\\hline
You ailing, death-awaiting moon, & Heavy, gloomy giant black moths \\
High upon heaven's dusty couch, & Massacred the sun's bright rays; \\
Your glance, so feverish overlarge, & Like a close-shut magic book \\
Lures me, like strange enchanting song. & Broods the distant sky in silence. \\
With unrequited pain of love & From the mists in deep recesses \\
You die, your longing deep concealed, & Rise up scents, destroying memory. \\
You ailing, death-awaiting moon, & Heavy, gloomy giant black moths \\
High upon heaven's dusty couch. & Massacred the sun's bright rays; \\
The lover, stirred by sharp desire & And from heaven earthward bound \\
Who reckless seeks for love's embrace & Downward sink with sombre pinions \\
Exults in your bright play of light & Unperceived, great hords of monsters \\
Your pale and pain-begotten flood, & On the hearts and souls of mankind... \\
You ailing, death-awaiting moon. & Heavy, gloomy giant black moths. \\\hline


9. Prayer to Pierrot & 10. Loot \\\hline
Pierrot! my laughter have I unlearnt!  & Ancient royalty's red rubies, \\
The picture's brightness dissolves. & Bloody drops of antique glory, \\
Black flies the standard now from my mast, & Slumber in the hollow coffins \\
Pierrot, my laughter have I unlearnt& Buried in the vaulted caverns, \\
O once more give me, healer of spirits, & Late at night with boon companions \\
Snowman of lyrics, monarch of moonshine, & Pierrot descends to ravish \\
Pierrot, my laughter! & Ancient royalty's red rubies. \\
&  Bloody drops of antique glory. \\
&  But there every hair a-bristle, \\
&  Livid fear turns them to statues; \\
&  Through the murky gloom, like eyes— \\
&  Glaring from the hollow coffins \\
&  Ancient royalty's red rubies. \\\hline

11. Red Mass & 12. Song of the Gallows \\\hline
To fearsome grim communion & The haggard harlot with scraggy gizzard \\
Where dazzling rays of gold gleam, & Will be his ultimate paramour. \\
And flickering light of candles, & Through all his thoughts there sticks like a gimlet \\
Comes to the alter Pierrot. & The haggard harlot with scraggy gizzard. \\
His hand, with grace invested, & Thin as a rake, round her neck a pigtail, \\
Rends through the priestly garments, & Joyfully will she embrace the rascal, \\
For fearsome grim communion & The haggard harlot! \\
Where dazzling rays of gold gleam. & \\
With signs of benediction & \\
He shows to frightened people & \\
The dripping crimson wafer: & \\
His heart—with bloody fingers & \\
In fearsome grim communion. & \\

\end{tabular}
\end{table}


\newpage
\begin{table}[h!]
\begin{tabular}{p{8.0cm}|p{8.0cm}} \hline

13. Decapitation & 14. The Crosses \\\hline
The moon, a polished scimitar & Holy crosses are the verses \\
Upon a black and silken cushion, & Where the poets bleed in silence, \\
So strangely large hangs menacing & Blinded by the peck of vultures \\
Through sorrow's gloomy night. & Flying round in ghostly rabble. \\
Pierrot wandering restlessly & On their bodies swords have feasted, \\
Stares upon high in anguished fear & Bathing in the scarlet bloodstream. \\
Of the moon, the polished scimitar & Holy crosses are the verses \\
Upon a black and silken cushion, & Where the poets bleed in silence. \\
Like leaves of aspen are his knees, & Death then comes; dispersed the ashes- \\
Swooning he falters, then collapses. & Far away the rabble's clamour, \\
He thinks: the hissing vengeful steel & Slowly sinks the sun's red splendour, \\
Upon his neck will fall in judgement, & Like a royal crown of glory. \\
The moon, a polished scimitar. & Holy crosses are the verses. \\\hline



15. Nostalgia & 16. Atrocity \\\hline
Sweetly plaintive is the sigh of crystal & Through the bald pate of Cassander, \\
That ascends from Italy's old players, & As he rends the air with screeches \\
Sadly mourning that Pierrot so modern & Bores Pierrot in feigning tender \\
And so sickly sentimental is now. & Fashion with a cranium driller. \\
And it echoes from his heart's waste desert, & He then presses with his finger \\
Muted tones which wind through all his senses, & Rare tobacco grown in Turkey \\
Sweetly plaintive, like a sigh of crystal & In the bald pate of Cassander, \\
That ascends from Italy's old players. & As he rends the air with screeches. \\
Now abjures Pierrot the tragic manner, & Then screwing a cherry pipe stem \\
Through the pallid fires of lunar landscape & Right in through the polished surface, \\
Through the foaming light-flood & Sits at ease and smokes and puffs the \\
mounts the longing, & Rare tobacco grown in Turkey \\
Surging high towards his native heaven. & From the bald pate of Cassander. \\
Sweetly plaintive, like a sigh of crystal. &\\\hline


17. Parody & 18. The Moonfleck \\\hline
Knitting needles, bright and polished, & With a snowy fleck of shining moonlight \\
Set in her greying hair,& On the shoulder of his black silk frock-coat \\
Sits the Duenna, mumbling,& So walks out Pierrot this languid evening. \\
In crimson costume clad. & Seeking everywhere for love's adventure. \\
She lingers in the arbour, & But what! something wrong with his appearance? \\
She loves Pierrot with passion, & He looks round and round and then he finds it- \\
Knitting needles, bright and polished, & Just a snowy fleck of shining moonlight \\
Set in her greying hair, & On the shoulder of his black silk frock-coat. \\
But, listen, what a whisper, & Wait now (thinks he) 'tis a piece of plaster, \\
A zephyr titters softly; & Wipes and wipes, yet cannot make it vanish. \\
The moon, the wicked mocker, & So he goes on poisoned with his fancy, \\
Now mimics with light rays & Rubs and rubs until the early morning \\
Bright needles, spick and span. & Just a snowy fleck of shining moonlight. \\

\end{tabular}
\end{table}


\newpage
\begin{table}[h!]
\begin{tabular}{p{8.0cm}|p{8.0cm}} \hline
19. Serenade & 20. Journey Home \\\hline
With a giant bow grotesquely & The moonbeam is the rudder, \\
Scrapes Pierrot on his viola; & Nenuphar searves as boat \\
Like a stork on one leg standing & On which Pierrot goes southward, \\
Sadly plucks a pizzicato. & The wind behind his sails, \\
Now here comes Cassander fuming & In deep tones hums the river \\
At this night-time virtuoso. & And rocks the light canoe, \\
With a giant bow grotesquely & The moonbeam is the rudder, \\
Scrapes Pierrot on his viola; & Nenuphar serves as boat. \\
Casting then aside the viola, & To Bergamo, his homeland, \\
With his delicate left hand he & Pierrot returns once more. \\
Grips the bald pate by the collar-  & Soft gleams on the horizon \\
Dreamily he plays upon him & The orient green of dawn. \\
With a giant bow grotesquely. & The moonbeam is the rudder. \\\hline

21. 0 Ancient scent &  \\
O ancient scent from far-off days,&  \\
Intoxicate once more my senses!&  \\
A merry swarm of idle thoughts&  \\
Pervades the gentle air.&  \\
A happy whim makes me aspire&  \\
To joys which I too long neglected.&  \\
O ancient scent from far-off days&  \\
Intoxicate me again.&  \\
Now all my sorrow is dispelled,&  \\
And from my sun-encircled casement&  \\
I view again the lovely world&  \\
And dream beyond the fair horizon.&  \\
O ancient scent from far-off days!&  \\

\end{tabular}
\caption{Pierrot Lunaire, 1912, English tranlsation}
\label{tab:pierrottext}
\end{table}

