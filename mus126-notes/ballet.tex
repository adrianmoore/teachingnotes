
\chapter{Ballet Music}
\label{balletmusic}

%\section{Scores}

\section{Impressionism and Debussy: Jeux}
\begin{itemize}
\item Listen: \url{https://www.youtube.com/watch?v=eT9ZQEZQSXs}
\item Score: \url{http://imslp.org/wiki/Jeux_%28Debussy,_Claude%29}
\item Reading: \url{http://adrian-moore.staff.shef.ac.uk/teaching/mus126/debussy_jeux_eimert.pdf}
\end{itemize}

Claude Debussy (1864-1918) brings much of what we know of impressionism in painting to the musical score. Easy to say but perhaps more difficult to define. Debussy was friends with painters such as James Whistler (1834-1903). When Whistler moved to Paris in 1855 he rented a studio in the Latin Quarter (bohemianism). His later work is very impressionistic and adopts the title Nocturne used explicitly to depict the obscure.      
Debussy was also heavily influenced by writers of the time such as Paul Verlaine (1844-1896) and St\'ephane Mallarm\'e, both symbolist writers. Symbolism focused naturally upon symbols and as a result dislocated itself from reality. As a consequence themes revolved around spirituality and the darker parts of the imagination, easily rendered through symbols (such as ravens (Poe)). 

Mallarm\'e's poem from 1876 \textit{Pr\'elude \`a l'apr\`es-midi d'un faune} was taken by Debussy for his orchestral prelude of the same name, composed in 1894 (see table~\ref{tab:faune}). Mallarm\'e was obsessed with the musical in words and indeed wrote a sonnet to Wagner. 

\begin{figure}[H]
\centering
\includegraphics[scale=0.2]{manetmalarme}\caption{St\'ephane Mallarm\'e by Edouard Manet (1876)}
\label{fig:quatierlatin}
\end{figure}

\begin{table}[h!]
\begin{tabular}{|l|l|} \hline
Le Faune: & \\
Ces nymphes, je les veux perp\'etuer. & These nymphs that I would perpetuate: \\
Si clair, & so clear \\
Leur incarnat l\'eger, qu'il voltige dans l'air & And light, their carnation, that it floats in the air  \\
Assoupi de sommeils touffus. & Heavy with leafy slumbers. \\
Aimai-je un r\^eve? & Did I love a dream? \\
Mon doute, amas de nuit ancienne, s'ach\`eve & My doubt, night’s ancient hoard, pursues its theme \\
En maint rameau subtil, qui, demeur\'e les vrais & In branching labyrinths, which being still \\
Bois m\^eme, prouve, h\'elas! que bien seul je m'offrais & The veritable woods themselves, alas, reveal \\
Pour triomphe la faute id\'eale de roses. & My triumph as the ideal fault of roses. \\
R\'efl\'echissons… & Consider… \\
ou si les femmes dont tu gloses & if the women of your glosses \\
Figurent un souhait de tes sens fabuleux! & Are phantoms of your fabulous desires! \\
Faune, l'illusion s'\'echappe des yeux bleus & Faun, the illusion flees from the cold, blue eyes \\
Et froids, comme une source en pleurs, de la plus chaste: & Of the chaster nymph like a fountain gushing tears: \\
Mais, l'autre tout soupirs, dis-tu qu'elle contraste & But the other, all in sighs, you say, compares \\
Comme brise du jour chaude dans ta toison? & To a hot wind through the fleece that blows at noon? \\
Que non! par l'immobile et lasse p\^amoison & No! through the motionless and weary swoon \\
Suffoquant de chaleurs le matin frais s'il lutte, & Of stifling heat that suffocates the morning, \\
Ne murmure point d'eau que ne verse ma fl\^ute & Save from my flute, no waters murmuring \\
Au bosquet arros\'e d'accords; et le seul vent & In harmony flow out into the groves; \\
Hors des deux tuyaux prompt \`a s'exhaler avant & And the only wind on the horizon no ripple moves, \\
Qu'il disperse le son dans une pluie aride, & Exhaled from my twin pipes and swift to drain \\
C'est, \`a l'horizon pas remu\'e d'une ride & The melody in arid drifts of rain, \\
Le visible et serein souffle artificiel & Is the visible, serene and fictive air \\
De l'inspiration, qui regagne le ciel. & Of inspiration rising as if in prayer. \\

\hline
\end{tabular}
\caption{Opening of \textit{Pr\'elude \`a l'apr\`es-midi d'un faune}}
\label{tab:faune}
\end{table}

Debussy achieved success early into his student days. He entered the Paris Conservatoire in 1873 for 11 years of study. It was here that Debussy first heard Wagner's early operatic output. He continued to study with C\'esar Franck and Ernest Guiraud. He was awarded the Premier Grand Prix de Rome in 1884 for his cantata \textit{L'Enfant prodigue} (the prodigal son). Debussy was 21. You can already hear the development of phrases and the growth of romantic harmonies. But this was a hasty work, written in a month and apparently in the style of Massenet so as to please the judges of the Prix de Rome. 

As winner of the prize (which continues to this day), Debussy set off for Rome in 1885. And clearly at this time Wagner remained a huge influence (though as an master, not necessarily as one to model). Around the turn of the decade, Wagner dominated French culture, particularly literary culture.

Both in literary circles and in music, symbolism highlighted the sensuous. Colour took on greater significance. With the darkness of war (Franco-Prussian and ultimately WWI), artists wanted to re-inject colour into the world. But equally at that time, the Russian nationalist composers were beginning to influence European composers. The great exhibitions (such as the World Fair of 1889) saw Rimsky-Korsakov conduct two concerts of Russian music. Out of this melting pot arose Debussy's early works. 

Debussy also knew Brahms and had visited and dined with him on a number of occasions. But one of his more substantial friendly relationships was with composer Erik Satie.

\subsection{Works}
Although you will probably know Debussy best by his ballet works, it is his opera \textit{Pell\'eas et M\'elisande} that should be heard more frequently. It occupied Debussy's 30s for around ten years. \textit{Pell\'eas et M\'elisande} is a play by Maurice Maeterlink (1862-1949) written around 1893. Debussy's opera received its premiere in 1902. 

And after Nocturnes (1899) subtitled: \textit{Nuages} (clouds), \textit{F\^etes} (Festivals) and \textit{Sir\'enes} (Sirens) Debussy naturally turned to the sea with \textit{La Mer} (1903-05). We know of Debussy's intent through letters to his publisher Auguste Durand. Debussy's personal life was one of turmoil at the time too. He left his first wife, Lily, who shot herself in the heart (and survived) and took up with one Mme Emma Barda, the wife of a wealthy banker. 

Although Debussy's orchestral colour is unsurpassed, to understand his musical syntax one must go to the piano pieces. These are extraordinarily difficult to play but Debussy wrote pieces of varying technical difficulty and the `Children's Corner' pieces are excellent works to study. The first volume of \textit{Images pour piano} (1904-05) deliver that archetypal impressionistic sound that one might imagine when looking at the works of Monet (The Water-Lily Pond of 1899). And indeed are laced with titles such as \textit{Reflets dans l'eau}. The second set of \href{http://petrucci.mus.auth.gr/imglnks/usimg/4/4a/IMSLP254485-PMLP02391-Debussy__Claude-Images_2e_Serie_pour_Piano_seul_Durand_6994_scan.pdf}{\textit{Images}} is more complex, and is set over three staves. \textit{Cloches à travers les feuilles} (Bells through the leaves - and clear use of the whole-tone scale), \textit{Et la lune descend sur le temple qui fut} And the moon descends on the temple that was, and \textit{Poissons d'or} Golden fish. 

But it is the later works where Debussy is really experimental. The second set of Pr\`eludes becomes ever more dissonant. And in Debussy's orchestral ballet masterpiece, \textit{Jeux} (1912) we hear very detailed construction. 

\subsection{Ballets Russes and Sergei Diaghilev}
This was a time of great impresarios (Sergei Diaghilev 1872 - 1929), great choreographers (Michel Fokine 1880-1942) and even greater ballet dancers (Vaslav Nijinsky 1889 - 1950). It was clear that around 1910, ballet's maestros were no less self-absorbed. Stravinsky and Debussy were often misunderstood by Fokine and Benois (designer) who were much more involved with the day-to-day running of the company (a company which worked and toured extremely hard). 

The first dress rehearsals for \textit{Jeux} were agonising. Costumes designed by L\'eon Bakst were inappropriate and Diaghilev altered them himself. But Diaghilev was working on \textit{Jeux} and \textit{The Rite of Spring} at the same time. The opening night was 15th May 1913. The ballet is your typical `fumble in the dark' - or as outlined to the audience at the premiere:

\begin{quotation}
The scene is a garden at dusk; a tennis ball has been lost; a boy and two girls are searching for it. The artificial light of the large electric lamps shedding fantastic rays about them suggests the idea of childish games: they play hide and seek, they try to catch one another, they quarrel, they sulk without cause. The night is warm, the sky is bathed in pale light; they embrace. But the spell is broken by another tennis ball thrown in mischievously by an unknown hand. Surprised and alarmed, the boy and girls disappear into the nocturnal depths of the garden.
\end{quotation}

Nijinsky was disheartened at the reaction to \textit{Jeux} which was not riotous but lacklustre. Debussy, writing to a friend shortly after the premiere was dissmayed at Nijinsky's approach which bordered (negatively) on the Dalcrozian (eurhythmics - approaching music through movement).

%%continue with ballet stuff and the rite
%%then the music of each ballet
%%emphasise the importance of the Eimert text

\section{The power of the new}
\begin{itemize}
\item \href{http://www.nytimes.com/2012/09/19/arts/music/radical-music-sometimes-shocking-sometimes-subtle.html?pagewanted=all&_r=0}{NY Times article}
\end{itemize}

\section{Modernism and Stravinsky: The Rite of Spring}
\begin{itemize}
\item Listen: \url{https://www.youtube.com/watch?v=rq1q6u3mLSM}
\item Listen: \url{https://www.youtube.com/watch?v=ewOBXph0hP4}
\item Score: \url{http://imslp.org/wiki/The_Rite_of_Spring_%28Stravinsky,_Igor%29}
\item Reading: Hill, P. (2000). Stravinsky: \textit{The Rite of Spring.} Cambridge Music Handbooks. Cambridge University Press

\end{itemize}


